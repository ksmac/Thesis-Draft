\documentclass[12pt,a4paper]{report}
\usepackage{graphicx}
\usepackage[letterpaper, portrait, left=2cm, right=2cm, top=2.5cm, bottom=2.5cm]{geometry}
\usepackage[labelfont=bf, hypcap=false]{caption}
\usepackage{subcaption}
\usepackage{placeins}
\usepackage{float}
\usepackage{url}
\usepackage[dvipsnames, table]{xcolor}
\usepackage{sectsty}
\usepackage[square,numbers]{natbib}
\usepackage[Conny]{fncychap}
\usepackage{lipsum}
\usepackage[toc,page]{appendix}
\usepackage{hyperref}
\usepackage{multirow}
\usepackage{mathrsfs,amsmath,amssymb}
\usepackage{esvect}
\usepackage{physics}
\usepackage[english]{babel}
\usepackage{cancel}
\usepackage{listings}
\usepackage{anyfontsize}
\newcommand{\Lagr}{\mathcal{L}}
\graphicspath{{images/}}
\usepackage{collcell}
\usepackage{hhline}
\usepackage{pgf}
\usepackage{colortbl}
\usepackage{tabu}
\usepackage[T1]{fontenc}
\usepackage[latin9,utf8]{inputenc}
\usepackage{pifont}
\usepackage{animate}
\usepackage{fancyhdr}
\usepackage{booktabs}
\usepackage{lipsum}% just to generate text for the example


\newcommand\items{3}
\newcommand*\rot{\rotatebox{90}}


\ChTitleVar{\centering\Huge\scshape\color{RoyalBlue}}
\ChNameVar{\centering\Huge\fontfamily{pbk}\selectfont\scshape\color{RoyalBlue}}

\makeatletter
\renewcommand{\DOCH}{%
    \vspace{-2cm}
	\color{ForestGreen}\mghrulefill{3\RW}\par\nobreak
	\vskip -0.5\baselineskip
	\mghrulefill{\RW}\par\nobreak
	\CNV\FmN{\@chapapp}\space \CNoV\thechapter
	\par\nobreak
	\vskip -0.5\baselineskip
}
\makeatother

\begin{document}
\begin{titlepage}


				\centering

				%\vspace*{\stretch{1}}
				{\Large\fontfamily{pbk}\selectfont Signal and background discrimination in the top squark production with   $\text{1\textit{l}} + \cancel{\it{E}}_{T} $ final states using XGBoost and Data Visualisation \par}
				\vspace{1.0cm}
			%$ \Tilde{t}\Tilde{t^*} \rightarrow t \Bar{t} \Tilde{\chi^0_1}\Tilde{\chi^0_1} \rightarrow b\Bar{b}l^{+}jj\cancel{\it{E}}_{T} $	     
			    \includegraphics[width=0.5\textwidth]{title.png}\par\vspace{0.5cm}
				%\vspace{1cm}
				{\Large A thesis presented for the Bachelor of Science (Honours) in the \par School of Physics and Astronomy, Monash University}\\
				\vspace{0.5cm}
				{\large\itshape Kenji Sato Macfarlane\\}
				\vfill
				Supervised by:\par
				Dr. Ursula Laa \par 
				Co-supervised by:\par
				Professor German Valencia \\
				
				\vspace{0.5cm}
				
			
			\includegraphics[width=0.15\textwidth]{monashlogo.png}\par\vspace{0.25cm}
				
				{\large \today\par}

\end{titlepage}
%-------------------------------------------------------------------------%

\begin{abstract}
\thispagestyle{plain}
\pagenumbering{roman}
\setcounter{page}{1}
    \noindent The Standard Model of particle physics well-describes the fundamental particles and their interactions, however, it fails to provide solutions to problems such as dark matter. Supersymmetric extensions to the Standard Model is one attempt to explain physics beyond the Standard Model, with experimental efforts for searches posing many difficulties. We studied the signature of the simplified model for the top squark decay occurring in supersymmetric extensions to the Standard Model; $pp \rightarrow \Tilde{t}\Tilde{t^*} \rightarrow t \Bar{t} \Tilde{\chi^0_1}\Tilde{\chi^0_1} \rightarrow b\Bar{b}jjl\cancel{\it{E}}_{T}$. To study the signature, we simulated the process and used xgboost to discriminate the top quark decays to the signal. We then visualise the data in a new approach using a guided tour, a form of data projection, to study the signature of the process. We found that, given the obvious difference in missing energy, the difference in the azimuthal component of the detected charged lepton and the measured missing energy are all important factors in separating the signatures of the top squarks to the top quark. \\
    
    \noindent\textbf{Cover figure:} A projection of one of the simulated data given by a guided tour.
    
    
    %We performed a statistical analysis including machine learning, in which we observed low AMS values of up to 2.8. Visualising the data in a guided tour, we observe that the $\cancel{\it{E}}_{T}$, the $\phi$ component of the $\cancel{\it{E}}_{T}$ and charged lepton, the $p_T$ of the lepton and the hadronic energy $H_T$ had strong contributions to the result.
\end{abstract}
\clearpage

\pagenumbering{roman}
\setcounter{page}{2}
\vspace*{8cm}
\subsection*{\centering\textit{Dedication}}
\textit{This work is dedicated to the memory of my father, James Macfarlane, who has inspired me to pursue knowledge, igniting my passion for science, taught me how to be a good person, and always believed in my ability and efforts to succeed in whatever I chose to do. You are gone but your belief in me has made this journey possible. Thank you for your love and support, Dad.} \\

\centerline{\textit{}} 
\newpage

\vspace*{4cm}
\section*{Acknowledgements}
\thispagestyle{plain}

I would like to thank; \\

\noindent Dr. Ursula Laa and Professor German Valenica for taking me on as their student and giving me guidance throughout the year with many valuable discussions and feedback. Without their help, I certainly wouldn't have been able to get through and enjoy working on this project. \\

\noindent Professor Di Cook of the Department of Econometrics and Business Statistics for her help in understanding the use of machine learning, statistical techniques, and data visualisation for high-dimensional data. \\

\noindent The Monash HEP group for encouraging us Honours students and providing us with a platform to learn particle physics not limited to our research area. \\

\noindent My fellow Honours cohort for going through this battlefield together, supporting each other through the late nights with many ideas and laughter.  \\

\noindent My friends for being supportive and the good times that helped me maintain sanity. I would also like to thank the Macfarlane family for being understanding in my desire to do Honours, and providing support when times were most difficult. 





\clearpage
%-------------------------------------------------------------------------%
\pagenumbering{roman}
\setcounter{page}{4}

\makeatletter
\renewcommand\tableofcontents{%
    \if@twocolumn
      \@restonecoltrue\onecolumn
    \else
      \@restonecolfalse
    \fi
   \newgeometry{top=0.6cm, bottom=2.5cm, left=2cm, right=2cm}
    \chapter*{\contentsname
        \@mkboth{%
           \MakeUppercase\contentsname}{\MakeUppercase\contentsname}}%
    \vspace{-2cm}
    \@starttoc{toc}%
    \if@restonecol\twocolumn\fi
    \restoregeometry
    }
\makeatother

\pagestyle{empty}
\tableofcontents
\clearpage

%-------------------------------------------------------------------------%
\pagestyle{fancy}
\fancyhead[L]{\empty}
\fancyhead[R]{\footnotesize
  \begin{tabular}[b]{@{}r@{}}
    \nouppercase{\leftmark}\\[3pt]
    \nouppercase{\rightmark}
  \end{tabular}%
}
\setlength{\headheight}{27pt} % check the log to be sure what this length should be
%-------------------------------------------------------------------------%

\pagenumbering{arabic}
\setcounter{page}{1}

%---------------------------------------------------------------------------%
\chapter{Introduction}
The discovery of the Higgs boson in 2012 at the Large Hadron Collider \cite{chatrchyan2012observation,aad2012observation} completed the Standard Model of particle physics, with fundamental particles and the force carriers mediating interactions of these particles. These particles and their interactions are now studied with high precision, however, properties of the universe unexplained by the Standard Model exits, one of which is dark matter. Regular matter in the universe constitutes of only 5\% of the total density, whereas dark matter constitutes of roughly 23\% \cite{thomson2013modern}. This variation alone suggests that we only understand a fraction of the physics that explains our universe.  \\

Many extensions to the Standard Model exist to explain physics beyond the Standarad Model. Supersymmetry is a concept that is included in many models attempting to explain new physics, one of which is the Minimal Supersymmetric Standard Model \cite{martin1997supersymmetry}. This model introduces superpartners to the Standard Model particle content in addition to extra Higgs, such that it looks `exactly duplicated', conveniently including new particles that satisfy properties assumed for dark matter. The mixture of the superpartners of the neutral Standard Model force carriers provide us with four \textit{neutralinos} in which the lightest one resembles the properties of dark matter. This lightest neutralino is thought to be the lightest supersymmetric particle and is stable \cite{martin1997supersymmetry}. In this project, we considered a simplified model from amongst many candidate decays in the Minimal Supersymmetric Standard Model. The decay process of interest is $\Tilde{t}\rightarrow t\Tilde{\chi}_1^0$ where the supersymmetric top squark ($\Tilde{t}$) produces the Standard Model top quark ($t$) and the lightest neutralino ($\Tilde{\chi}_1^0$), as the top squarks are thought to be the next-lightest supersymmetric particle. \\

The success of the Higgs discovery is credited to theory, and statistical techniques and analysis tools used in collider experiments. Searches for supersymmetric particles have followed the same procedures, where experiments have been gradually pushing the limits to for masses to the top squarks and neutralinos. Simulations and Machine Learning are some key aspects in analyzing experimental data and so we follow similar techniques in our method. Furthermore, we can visualize the data with a guided tour in hopes to get an insight into the physics enfolding with these particles. \\

\newpage
\noindent\textbf{Outline:}\\
We present a brief introduction to the Standard Model and the Minimal Supersymmetric Standard Model in Chapter \ref{chap:2}, followed by the details of the top squark, its production, and how searches for this particle are performed experiments in Chapter \ref{chap:3}. Chapter \ref{chap:4} presents the tools and supporting arguments related to our method, particularly in how particle detectors work and how simulations are performed to assist in searches. Our methods for simulation and the creation of the datasets are discussed in Chapter \ref{chap:5}. The method for analysis, particularly machine learning and the applied statistical techniques, are explored in Chapter \ref{chap:6}. We present our results in Chapter \ref{chap:7} with the performance our classifiers and data visualisation with a guided tour.


%-------------------------------------------------------------------------%
\chapter{The Standard Model and Beyond}
\label{chap:2}
%\section{The Standard Model and Beyond}

%-------------------------------------------------------------------------%
\section{The Standard Model}
The Standard Model (SM) is made up of three generations of spin-$1/2$ \textit{fermions} as fundamental particles, spin-1 \textit{gauge bosons} and the spin-0 \textit{Higgs boson}. The interactions of the fermionic matter are mediated by the gauge bosons which are well represented by Quantum Field Theory (QFT); these correspond to the three fundamental forces - electromagnetism, the weak force, and the strong force. Unfortunately, the remaining fundamental force, gravity, is yet to be understood at the quantum level. The Higgs, rather the Higgs field, is responsible for the particles acquiring their masses. \\

The fundamental particles are composed of \textit{leptons} and \textit{quarks}, shown in Table \ref{tab:SMFerm}. Each of the three generations of quarks has an up-type and a down-type quark categorized by their electric charges ($+2/3 \text{ or } -1/3$), and the leptons are either charged or neutral, with the neutral components are known as \textit{neutrinos}. The up-type quarks are the \textit{up}, \textit{charm} and \textit{top} quarks, and the down-type quarks are the \textit{down}, \textit{strange} and \textit{bottom} quarks. These quarks exist in a bound state known as \textit{hadrons}, except for the top quark that decay before \textit{hadronisation}, and can never be observed directly. Hadrons are categorized into mesons or (anti-)baryons; Mesons consist of two quarks - a quark-antiquark pair, and (anti-)baryons are made up of three (anti-)quarks. Proton and neutrons are some examples of baryons, and pions and kaons are mesons \cite{thomson2013modern}. These fundamental particles differ greatly in mass, with the first generation being the lightest and sequentially becoming heavier for the second and third generations. The first generation of elementary particles represents the basic building blocks of the low energy Universe \cite{thomson2013modern}. \\

\begin{table}[htbp]
    \centering
    \begin{tabular}{c||c|c|c|c}
    \toprule
    & $1^{\text{st}}$ Generation & $2^{\text{nd}}$ Generation & $3^{\text{rd}}$ Generation & Electric Charge \\
    \midrule
    \multirow{2}{1.2cm}{quarks} & $u$ & $c$ & $t$ & $+2/3$ \\
     & $d$ & $s$ & $b$ & $-1/3$ \\
    \midrule
    \multirow{2}{1.2cm}{leptons} & $e$ & $\mu$ & $\tau$ & $\mp1$ \\
     & $\nu_e$ & $\nu_\mu$ & $\nu_\tau$ & $0$ \\
    \bottomrule
    \end{tabular}
    \caption{The symbols of the three generation of spin-$1/2$ fermionic matter in the Standard Model.}
    \label{tab:SMFerm}
\end{table}


The SM is governed by a combination of complex gauge symmetries, $SU(3)_C\times SU(2)_L \times U(1)_Y$, with each symmetry described by its corresponding gauge bosons shown in Table \ref{tab:SMBos} \cite{thomson2013modern}. It comprises of four vector bosons in the form of \textit{gluons}, \textit{photons}, \textit{Z bosons} (neutral) and \textit{W bosons} (electrically charged). The gluons mediate the strong force, or strong interactions, described by Quantum Chromodynamics (QCD), binding quarks into hadrons for example. The three remaining gauge bosons are a result of the Higgs mechanism \cite{higgs1964broken, englert1964broken}. The Higgs mechanism involves the Higgs boson, rather the Higgs field - a doublet comprised of a neutral and a charged field component, that acquires some vacuum expectation value (vev). This produces a physical Higgs and three charged and neutral massless Goldstone bosons that are `eaten' by the gauge bosons through electroweak\footnote{The electroweak interaction describes both electromagnetism and the weak interaction as a unified force  ($SU(2)_L \times U(1)_Y$) in the weak scale \cite{thomson2013modern}.} symmetry breaking. The $SU(2)$ weak gauge bosons $W$ and the $U(1)$ gauge bosons $B$ through the Higgs mechanism provide terms that form the gauge bosons in Table \ref{tab:SMBos}; a mixture of $B$ and $W^0$ that become the massless photon and Z boson that is roughly 91GeV/c$^2$ \cite{tanabashi2018review}, and $W^1 \mp iW^2$ that become the W boson that is roughly 80GeV/c$^2$ \cite{tanabashi2018review}. These terms will be relevant again when we discuss a supersymmetric model. Before considering supersymmetric models, there is an extended theory to the SM known as the Two-Higgs-doublet model (2HDM).  The Higgs sector in the 2HDM has two doublets opposed to the one in SM; the up-type Higgs ($H_u$) and the down-type Higgs ($H_d$) that contain both neutral and charged components \cite{2HDM}. In the 2HDM, the number of degrees of freedom doubles to eight, leading to 5 massless scalars including a SM-like Higgs boson, shown in Table \ref{tab:SMBos}.  \\

\begin{table}[htbp]
    \centering
    {\renewcommand{\arraystretch}{1.2}
    \begin{tabular}{c||c|c}
    \toprule
    Model & Gauge Bosons  & Scalar Bosons \\
    \midrule
    Standard Model & $g$, $\gamma$, $Z^0$, $W^\pm$ & $h^0$ \\
    \hline
    2HDM & $g$, $\gamma$, $Z^0$, $W^\pm$ & $h^0$, $H^0$, $A^0$, $H^\pm$ \\
    \bottomrule
    \end{tabular}}
    \caption{The symbols of the force carriers in the SM and the 2HDM extension. The three extra Higgs in the 2HDM, $H^0$, $A^0$, and $H^\pm$, come from the extra Higgs field introduced in the model.}
    \label{tab:SMBos}
\end{table}


%-------------------------------------------------------------------------%
\subsection{The top quark and its decay}
Amongst the SM particles, the top quarks and their decays will correspond to our background events. The top quark has a rest mass of $m_t\approx175\text{GeV/c}^2$ (we will express mass and energy in natural units $c=1$ hereon) and an electric charge of +2/3$e$ \cite{tanabashi2018review} making it the only known elementary particle in the SM to be heavier than the Higgs boson. The top quark is produced predominantly through gluon-gluon fusion and it almost always decays through the weak interaction into a pair of bottom quarks and W bosons. Through the weak interaction, the W boson decays into either a charged lepton-neutrino pair or a pair of quarks and quarks that form hadronic jets. The possible decay modes are given in Figure \ref{fig:topdecay}. \\

%The theoretical predictions of top quarks and the bottom quark was established in 1973 by Kobayashi and Masakawa to explain charge-parity (CP) violation in some meson decays \cite{griffiths2008introduction, kobayashi1973cp}. Experimental evidence for the bottom quark, the generationally paired particle to the top quark, was found at Fermilab in 1977 \cite{herb1977observation}. Nearly two decades later in 1994, the top quark was discovered at the Tevatron with a centre-of-mass energy\footnote{The Centre-of-Mass energy, denoted as $\sqrt{s}$, is a Lorentz invariant value such that $s=\Big( \sum\limits_{i=1}^2 E_i \Big)^2 - \Big( \sum\limits_{i=1}^2 \overrightarrow{p_i} \Big)^2 $. Note that $c=1$ and each $i$ represents the protons in the beam.} of $\sqrt{s}=1.8\text{TeV}$ through the process $p\Bar{p} \rightarrow t\Bar{t}$ \cite{abachi1994search, coll1994evidence, abachi1995observation}. 

\begin{figure}[htbp]
    \centering
    \begin{minipage}{0.24\linewidth}
        \includegraphics[width=\linewidth]{top1.png}
        \label{fig:top1}
    \end{minipage}
    \begin{minipage}{0.24\linewidth}
        \includegraphics[width=\linewidth]{top2.png}
        \label{fig:anttop1}
    \end{minipage}
    \begin{minipage}{0.24\linewidth}
        \includegraphics[width=\linewidth]{top3.png}
        \label{fig:top2}
    \end{minipage}
    \begin{minipage}{0.24\linewidth}
        \includegraphics[width=\linewidth]{top4.png}
        \label{fig:anttop2}
    \end{minipage}
    \caption{Feynman diagrams of possible (anti-)top decays. From left to right, we see the top decay with leptonic final states, top decay with hadronic final states, anti-top decay with leptonic final states and anti-top decay with hadronic final states.}
    \label{fig:topdecay}
\end{figure}

%-------------------------------------------------------------------------%
\section{The Minimal Supersymmetric Standard Model}
The Minimal Supersymmetric Standard Model (MSSM) is one of many extensions to the SM that helps to explain new physics beyond the SM. The MSSM particle content duplicates the SM particle content thus keeping extra particles introduced to its minimum. A supersymmetric transformation that yields the same quantum numbers as that of the SM retains the symmetry groups of the SM discussed in the preceding subsection \cite{aitchison2007supersymmetry}. The fermions and bosons in the SM shown in Tables \ref{tab:SMFerm} and \ref{tab:SMBos}  have \textit{superpartners}; the supersymmetric counterparts to each particle in the SM as seen in Tables \ref{tab:SUSYspart} and \ref{tab:SUSYinos}. By convention, the SM particles and their superpartners are distinguished by a tilde. \\

The superpartners to the SM fermions are called \textit{squarks}, \textit{sleptons} and \textit{sneutrinos} whereas the superpartners to the bosons in the SM end with an ``-ino" e.g. \textit{gaugino}. The `symmetry' in supersymmetry is the symmetry between fermions and bosons, meaning that the fermions in the SM are bosons in the MSSM, and vice versa \cite{martin1997supersymmetry}. Tables \ref{tab:SUSYspart} and \ref{tab:SUSYinos} illustrate that the number of new particles introduced is kept to a \textit{minimum} \cite{aitchison2007supersymmetry}. This symmetry, however, would require that these MSSM particles have the same masses as their SM counterparts. We know that this is not true because such particles have not been observed so far in collider experiments. The symmetry must be broken at some high energy scale in an unknown way, allowing these particles to acquire much heavier masses than their SM counterparts that are unobservable with current collider experiments' energies. \\

\begin{table}[htbp]
    \centering
    \begin{tabular}{c||c|c|c}
    \toprule
    & $1^{\text{st}}$ Generation & $2^{\text{nd}}$ Generation & $3^{\text{rd}}$ Generation \\
    \midrule
    & \\[-2.7ex]
    \multirow{2}{1.4cm}{squarks} & $\Tilde{u}$ & $\Tilde{c}$ & \small$\Tilde{t}$ \\
     & $\Tilde{d}$ & $\Tilde{s}$ & $\Tilde{b}$ \\
    \midrule
    
    \multirow{2}{1.4cm}{sleptons} & $\Tilde{e}$ & $\Tilde{\mu}$ & $\Tilde{\tau}$ \\
     & $\Tilde{\nu_e}$ & $\Tilde{\nu_\mu}$ & $\Tilde{\nu_\tau}$ \\
    \bottomrule
    \end{tabular}
    \caption{The symbols for the MSSM squarks and sleptons. These are bosons with spin-0.}
    \label{tab:SUSYspart}
\end{table}

\begin{table}[htbp]
    \centering
    \begin{tabular}{c|c}
    \toprule
       Gauginos  & Higgsinos \\
       \midrule
        & \\[-2.5ex]
      $\Tilde{g}$, $\Tilde{B}$, $\Tilde{W}^0$, $\Tilde{W}^\pm$ & $\Tilde{H}_u$,  $\Tilde{H}_d$ \\
     \bottomrule
    \end{tabular}
    \caption{The symbols for the MSSM gauginos and Higgsinos. The gauginos includes the gluino, Bino, and Winos (neutral and charged). The Higgsinos is a superpartner to the 2HDM Higgs  field components (each with a neutral and charged component) listed instead.}
    \label{tab:SUSYinos}
\end{table}
%with three neutral components ($\Tilde{h}^0$, $\Tilde{H}^0$ and $\Tilde{A}^0$) and two charged components ($\Tilde{H}^\pm$).
In the MSSM, a new global symmetry \textit{R-parity} is required. This stems from the violation of fundamental quantum numbers in the SM known as baryon ($B$) and lepton ($L$) numbers, now unconserved in the MSSM. Strong experimental constraints for $B$- and $L$- violating processes support the need for a method for these numbers to conserve. R-parity is therefore assumed to be a conserved quantum number that resolves this issue, given by Equation (\ref{eq:RParity})
\begin{equation}
    P_R=(-1)^{3(B-L)+2s}
    \label{eq:RParity}
\end{equation}
where $s$ is the spin of the particle. This symmetry conveniently separates SM particles and their MSSM superpartners in a way that the SM particles have $P_R=+1$ (even R-parity), whereas the sparticles all have $P_R=-1$ (odd R-parity) \cite{martin1997supersymmetry}. In MSSM, the conservation of $R$-parity results in supersymmetric events as pair-produced events \cite{aitchison2007supersymmetry}. \\

With the new scalars in Table \ref{tab:SUSYinos}, the MSSM introduces dark matter candidates known as \textit{neutralinos}. These particles are a mixture of the neutral components of the MSSM gauginos (excluding the gluino) and Higgsinos as shown in Figure \ref{fig:SUSY}. The four neutralinos denoted $\Tilde{\chi}_i^0$ where $i=1,2,3,4$ have a mass hierarchy of $ m_{\Tilde{\chi}_1^0} < m_{\Tilde{\chi}_2^0} < m_{\Tilde{\chi}_3^0} < m_{\Tilde{\chi}_4^0}$ \cite{martin1997supersymmetry}. The lightest of the four, $\Tilde{\chi}_1^0$, is thought to be the lightest supersymmetric particle (LSP) which is stable, supporting the theoretical properties of a proposed dark matter in cosmology. This assumption holds only when \textit{R-parity} from Equation (\ref{eq:RParity}) is conserved \cite{martin1997supersymmetry}. This means that the two-body decay of $\Tilde{\chi}_i^0$ is pair-produced with other supersymmetric particle. The charged gauginos and Higgsinos mix to form the \textit{charginos}, shown in Figure \ref{fig:SUSY}, that will always decay. In collider experiments, the `detection', or measurement, of these neutralinos rely on the missing energy of reconstructed SUSY event, much like the SM neutrinos. \\
%charged($\Tilde{W}^\pm$ and $\Tilde{H}^\pm$)

\begin{figure}[htbp]
    \centering
    \includegraphics[width=\linewidth]{SMtoMSSM.png}
    \caption{A diagram depicting SM force carriers to their MSSM counterparts. The gluon is directly supersymmetric to the gluino, with the Higgs bosons and the remaining gauge bosons are supersymmetric to the Higgsinos and gauginos, respectively. The neutral components of the gauginos and Higgsinos mix to give rise to the neutralinos, likewise for the charged components to the charginos.}
    \label{fig:SUSY}
\end{figure}


%\footnote{The gauge invariance required in the SM ``accidentally" \cite{martin1997supersymmetry} guarantees the conservation of such quantum numbers in most interactions.}

%-------------------------------------------------------------------------%
\chapter{The top squark and its production}
\label{chap:3}
%\section{The stop ($ \Tilde{t} $) and its production}
%-------------------------------------------------------------------------%
\section{Possible decay channels of the top squarks} 
\label{sec:stopDecay}
Just as the SM particles and their decays are determined by symmetries, couplings, and masses of the particles involved, the decay of stops is also determined by parameters in the MSSM. The mass of stops would determine the favorable decay modes. The top squarks have two gauge eigenstates that mix strongly due to the large top quark mass. This results in mass eigenstates of two top squarks, the lighter of which is possibly lighter than other supersymmetric particles other than the neutralinos \cite{thomson2013modern, boehm2000decays}. The important assumption made in these decays is that the stops are heavier than the neutralino ($m_{\Tilde{t}} > m_{\Tilde{\chi}_1^0} $) so that the neutralinos is the LSP. \\

A heavy enough stop with a mass heavier than the sum of the top quark mass and neutralino mass ($m_{\Tilde{t}} > m_t+m_{\Tilde{\chi}_1^0}$) would undergo a two-body decay into a quark with neutralinos: $\Tilde{t}\rightarrow q\Tilde{\chi}_1^0$. In reference \cite{boehm2000decays}, the charm quark pair produced with the neutralino is considered the dominant process from a low-mass top squark and high-mass neutralino, but in this project we approach the simplified model that is the other possible decay in the MSSM; top squarks to top quark-neutralino pair $\Tilde{t}\rightarrow t\Tilde{\chi}_1^0$. A lighter stop that is heavier than the sum of the bottom quark mass, W-boson mass and neutralino mass ($m_{\Tilde{t}} > m_b + m_W + m_{\Tilde{\chi}_1^0}$), but the mass difference between the stops and neutralinos ($\Delta m = m_{\Tilde{t}} - m_{\Tilde{\chi}_1^0} $) is lighter than the top quark ($\Delta m < m_t$) allows a three-body decay with virtual W bosons in the process $\Tilde{t}\rightarrow b W \Tilde{\chi}_1^0$. Further decays are apparent when supersymmetric particles other than $\Tilde{\chi}_1^0 $ are lighter than the stop, for example, the charginos $\Tilde{\chi}_1^{\pm}$. If $\Tilde{\chi}_1^{\pm}$ is lighter than the stops, the decay results in a bottom quark-chargino pair: $\Tilde{t}\rightarrow b\Tilde{\chi}_1^+$ \cite{boehm2000decays}. \\

%would imply a three-body decay  Four-body decays into a combination of bottom quarks, neutralinos and SM fermions, may also be possible: $\Tilde{t}\rightarrow b\Tilde{\chi}_1^0 f \Bar{f}'$ \cite{boehm2000decays}. However, the four-body decay is only relevant when the two- and three-body decays are kinematically forbidden. This would also allow a flavor-suppressed decay to a charm quark: $\Tilde{t}\rightarrow c\Tilde{\chi}_1^0$ \cite{aad2014search}. \\

A diagram from \cite{aad2014search} is shown in Figure \ref{fig:decayMode}, reflecting the statements above under the assumption that $ \Tilde{\chi}_1^0 $ and $\Tilde{t}$\footnote{The SM quarks have a left- and right-handed component in which they are a doublet and a singlet respectively. The MSSM counterpart also has left- and right-handed components that mix to form two distinct squarks, hence the notation $\Tilde{t}_1$ for the lighter stops in Figure \ref{fig:decayMode}.} are the lightest and next-to-lightest particles in the MSSM. For this project, we stick to a simplified model with final states that include one charged lepton, some missing energy and some hadronic jets as shown in Figure \ref{fig:stopDecay}. The
desired background events are the process given by Equation (\ref{eq:background}) and the desired signal events are given by Equation (\ref{eq:signal}), in which the $b$-quarks originates from the decay $t\Bar{t} \rightarrow bW^+ \Bar{b}W^-$ and are detected as $b$-jets. \\

\begin{equation}
 pp \rightarrow t \Bar{t} \rightarrow b\Bar{b}jjl\cancel{\it{E}}_{T},
 \label{eq:background}
\end{equation}
\begin{equation}
  pp \rightarrow \Tilde{t}\Tilde{t^*} \rightarrow t \Bar{t} \Tilde{\chi^0_1}\Tilde{\chi^0_1} \rightarrow b\Bar{b}jjl\cancel{\it{E}}_{T},
  \label{eq:signal}
\end{equation}

%The doublet allows mixing with other quarks (e.g. tops with the bottom) thus making them more complicated and potentially heavier.
\begin{figure}[htbp]
    \centering
    \includegraphics[width=0.75\linewidth]{decaymodes.png}
    \caption{Possible decay modes for stops within the mass-parameter space of $\Tilde{t}_1 $ and $ \Tilde{\chi}_1^0 $ \cite{aad2014search}. The blue filled region is kinematically forbidden due to the neutralino mass being heavier than the stop mass. As the mass of stops get heavier, the allowed decays differ, favoring on-shell decays over off-shell decays.}
    \label{fig:decayMode}
\end{figure}


\begin{figure}[htbp]
    \centering
    \includegraphics[width=0.5\linewidth]{stop_decay2.png}
    \caption{An example decay of the signal of interest $\Tilde{t}\Tilde{t}^* \rightarrow t\bar{t}\Tilde{\chi}_1^0\Tilde{\chi}_1^0 $ with a final state of one charged lepton-neutrino pair, and hadronic jets originating from the quark-antiquark pairs and bottom quarks.}
    \label{fig:stopDecay}
\end{figure}


%-------------------------------------------------------------------------%
%\section{Mass of the top squark}
%\label{sec:stopMass}
%The mass eigenstates for the top squarks are given by the equation

%\begin{align}
%    \begin{pmatrix} \Tilde{t}_1 \\ \Tilde{t}_2 \end{pmatrix} = 
%    \begin{pmatrix} \cos\theta_\Tilde{t} & -\sin\theta^*_\Tilde{t} \\ \sin\theta_\Tilde{t} & \cos\theta^*__\Tilde{t} \end{pmatrix}
%    \begin{pmatrix} \Tilde{t}_L \\ \Tilde{t}_R \end{pmatrix}
%    \label{eq:stopMass}
%\end{align}
%where $ \theta_\Tilde{t} $ is the stop mixing angle in the range $ 0 \leq {\theta_\Tilde{t}} \leq \pi $ satisfying $ |\cos\theta_{\Tilde{t}}|^2 + |\sin\theta_{\Tilde{t}}|^2 = 1 $ \cite{martin1997supersymmetry}. \\

%The mass splitting of the two stops $ \tilde{t}_1 $ and $ \tilde{t}_2 $ arise from the squared-mass matrix for stops, where the off-diagonal elements involves a large top-quark Yukawa coupling ($y_t$ term) that induces such a phenomena \cite{kraml2016scalar}. Diagonalizing gives the $\Tilde{t}_L$ and $\Tilde{t}_R$ components on the right-hand side of Equation (\ref{eq:stopMass}). One possible model in the MSSM predicts that $\Tilde{t}_1$ is the lightest of all squarks predominantly theorized to be $\Tilde{t}_R$ which is the right-handed stops \cite{martin1997supersymmetry}. Although there has been no success in the direct detection of the particle, experimental efforts have been made to set constraints on its mass \cite{kraml2016scalar, aad2014search, abdughani2018probing, sirunyan2018search, yoshihara2017search}.

%-------------------------------------------------------------------------%
\section{Mass parameters for the top squarks and neutralinos}
Experiments at the Large Hadron Collider have set limits on the stop mass and neutralino mass since the operation began in 2008. Several searches have been made for new particles and their properties with statistical techniques to establish limits, and make a discovery in a fortunate scenario such as the Higgs boson. A frequentist approach, in short, is the main statistical technique taken where a numerical analysis is based on a probability-based hypothesis test.

%-------------------------------------------------------------------------%
\subsection{Frequentist statistics - setting limits in searches}
\label{sec:freqStat}
%A likelihood function provides us with the probability distribution function evaluated for the observed data, in which the extended likelihood function to account for the Poisson distribution of the number of events $N$ is given by
%\begin{equation}
%    L(\vec{x}_1,...,\vec{x}_n; \vec{\theta}) 
%    = \frac{e^{-\mu_n(\vec{\theta})}\mu_n(\vec{\theta})}{n!}\prod^n_{i=1} f(\vec{x}_i;\vec{\theta})
%    \label{eq:genLikelihood}
%\end{equation}
%where $\vec{x}_i$ is an individual observation in a set of $i=1,...,n$ random variables amongst $N$ number of uncorrelated events, $\mu_n(\vec{\theta})$ is the expected number of events dependant on some nuisance parameter $\vec{\theta}$, a parameter in which unknown parameters coupled to the parameter of interest from detector response is accounted for \cite{lista2017statistical}.
The hypothesis test most commonly used in collider experiments is a background-only (null) hypothesis $H_0$  $(b)$. The alternate hypothesis  $H_1$ then includes both signal and background events $(s+b)$. In the absence of systematic uncertainties, one can construct the likelihood ratio evaluated with likelihoods\footnote{Collider experiments are considered a form of counting experiment. The observed events assume a Poisson distribution with a mean of $\mu_s+\mu_b$ where $\mu_{s(b)}$ is the estimated number of signal(background) \cite{lista2017statistical, adam-bourdarios_learning_2014}, hence the likelihood is $P(n|\mu_s,\mu_b) = \frac{(\mu_s+\mu_b)^n}{n!}e^{-(\mu_s+\mu_b)}$} $L(\mu)$ \cite{LHCstats}; $H_0 \rightarrow L_{s+b}$ and $H_1 \rightarrow L_b$ given by
\begin{equation}
    \lambda(\mu) = \frac{L_{s+b}}{L_b}
    %(\mu, \hat{\hat{\theta}}), (\hat{\mu},\hat{\theta})
    \label{eq:genLR}
\end{equation}
%for some maximum-likelihood estimators\footnote{The maximum likelihood estimators provide the parameter values that maximize the search for a particular observation within the set.} $\hat{\mu}$ and $\hat{\theta}$, and $\hat{\hat{\theta}}$ being the conditional maximum-likelihood estimator \cite{cowan2011asymptotic}. \footnote{The signal discovery test statistic is $q_0$ where it holds the value $-2 \ln \lambda (0)$ for $\hat{\mu}=0$ and zero otherwise.}
where $\lambda$ is in the range $0 \leq \lambda \leq 1$. Equation (\ref{eq:genLR}) is then used in a test statistic $q_\mu$ defined as
\begin{equation}
    q_\mu = \left\{
        \begin{array}{ll}
            -2\ln \lambda(\mu) & \quad \hat{\mu} \leq \mu \\
            0 & \quad \hat{\mu} \geq \mu
        \end{array}
    \right.
\end{equation}
for some signature $\mu$ (0 for $H_0$ and 1 for $H_1$) and its effective estimator $\hat{\mu}$ that maximizes the likelihood. For a hypothesis assuming $\mu$, the incompatibility of the hypothesis with the data results in a large $q_\mu$. This is further quantified with a \textit{p-value} given by the probability distribution function (pdf) of the test statistic $q_\mu$, $f(q_\mu|\mu)$, integrated over the tail of its distribution with some observation \cite{cowan2011asymptotic}, expressed as \\
\begin{equation}
    p_\mu = \int\limits^\infty_{q_{\mu,obs}} f(q_\mu|\mu) dq_\mu
    \label{eq:p-value}
\end{equation}

In placing limits, physicists use the confidence level $CL_\mu$ that inverts the integral in Equation (\ref{eq:p-value}) to give
\begin{equation}
    CL_\mu = \int\limits_{-\infty}^{q_{\mu,obs}} f(q_\mu|\mu) dq_\mu
    \label{eq:CLmu}
\end{equation}
where $\mu$ is testing for the background $b$ or the signal plus background $s+b$ case, instead of background versus signal. The pdf of both $b$ and $s+b$ overlap when the number of expected signal is extremely low, supporting neither hypothesis.
The signal confidence level $CL_s$ prevents accidental exclusions from the incorrect inference of p-values to such cases, as $CL_s$ lacks frequentist coverage in parameter regions where experiments are insensitive to the expected signal \cite{LHCstats}. The $CL_s$ is given as a ratio of the p-values of the two hypotheses:
\begin{equation}
    CL_s = \frac{CL_{s+b}}{CL_b} \rightarrow \frac{p_{s+b}}{1-p_b}
    \label{eq:CLS}
\end{equation}

%This allows us to compute the significance\footnote{For a discovery to be claimed, particle physicists require a minimum of $Z=5$ that corresponds to $p=2.87\times10^{-7}$ when rejecting the background only hypothesis.} $Z$ and its associated $p$-value with the relationship
%\begin{equation}
%    Z = \Phi^{-1}(1-p) = \sqrt{q_\mu}
%    \label{eq:Z}
%\end{equation}
%The exclusion of the hypothesis $\mu$ is supported when a threshold $p=\alpha$ is applied with a low value e.g. $\alpha=0.05$ for a 95\% confidence level ($Z=1.64$) \cite{cowan2011asymptotic}. \\

%Collider experiments are considered a form of counting experiments, thus the observed events are categorized under a Poisson distribution with a mean of $\mu_s+\mu_b$ where $\mu_{s(b)}$ is the estimated number of signal(background) \cite{lista2017statistical, adam-bourdarios_learning_2014}. This gives us a more simplified version to Equation (\ref{eq:genLikelihood});
%\begin{equation}
%    P(n|\mu_s,\mu_b) = \frac{(\mu_s+\mu_b)^n}{n!}e^{-(\mu_s+\mu_b)}
%    \label{eq:Poisson}
%\end{equation}
%whose likelihood ratio is given by
%\begin{equation}
%    \lambda = \frac{P(n|\mu_s,\mu_b)}{P(n|\hat{\mu_s},\mu_b)}
%    \label{eq:likelihoodRatio}
%\end{equation}
%where $\hat{\mu_s}$ is the maximum likelihood estimator of $\mu_s$. \\


%-------------------------------------------------------------------------%
\subsection{Exclusion limits for the stop and neutralino masses}
The exclusion of masses for new particles is performed through statistical analysis of the results. The exclusion of stop and neutralino masses is visually presented as an exclusion curve in CMS and ATLAS, such as those seen in Figure \ref{fig:limits} \cite{cms2019search} where \textit{inside} the expected limit (in red) and observed limit (in black) have already been excluded. \\

\begin{figure}[htbp]
    \centering
    \includegraphics[width=12cm, height= 12cm]{stop_limits.png}
    \caption{The latest cross-section and mass limits for the process $\Tilde{t}\Tilde{t}^* \rightarrow t\bar{t}\Tilde{\chi}_1^0\Tilde{\chi}_1^0 $ set by the CMS experiment \cite{cms2019search}. Inside the black curve are the excluded masses for stops and neutralino, and outside this line remains a  region for allowed masses for these particles, with the stop mass' limit pushed to the TeV scale. According to the colour bar, the further out into to parameter space we travel to, the allowed cross-section is extremely small at the order of $10^{-2}$pb.}
    \label{fig:limits}
\end{figure}

In this particular figure, it is presented that the expected limit to the process $pp \rightarrow \tilde{t}\tilde{t}^* \rightarrow t\tilde{\chi}_1^0$ falls under that of the observed limit when both masses are high i.e. when stop masses are in the order of 1-1.2TeV and neutralino masses in the order of $\sim600$ GeV. It is also shown that the observed limit falls under the expected limit when there is a massive mass difference between the stop and the neutralino i.e. $m_{\tilde{\chi}_1^0}\ll m_{\tilde{t}}$. Furthermore, the upper limit on the cross-section for the simplified model is provided in colour-code, such that  cross-section values above the corresponding colour will be excluded. This suggests that our signal selection efficiency is high when searching in regions with lower
limits. \\

The masses around $ m_{\tilde{t}} \approx 1.2$ TeV are an interesting range of parameters to follow, mainly to observe how the mass difference $\Delta m$ affects the performance of our machine learning classifiers. Intuitively, the large difference implies that the final states are highly energetic compared to the background events, and hence more easily differentiated than when $\Delta m$ is small. In addition, reference \cite{roxlo2018opening} has explored the performance of their neural networks discriminating stop production to top production, albeit in a dilepton final state within the $\tilde{t} \rightarrow t \tilde{\chi}_1^0$ decay. To do a cross-check, their choice of mass parameters are our fourth benchmark point: $m_{\tilde{t}} =750$ GeV and $m_{\tilde{\chi}_1^0} = 1$ GeV, well within the exclusion limit in Figure \ref{fig:limits}. Table \ref{tab:benchmarks} shows a summary of mass parameters chosen to explore. These parameters were chosen particularly to observe the effect of varying $\Delta m$, with a slight change in stop masses to ensure we are choosing masses just outside of the exclusion curve.

\begin{table}[htbp]
    \centering
    \begin{tabular}{c|c|c|c} 
    \toprule
    Benchmark No. & Position & $m_{\Tilde{t}}$ (TeV) & $m_{\Tilde{\chi}_1^0}$ (GeV) \\
    \midrule
    \rowcolor{gray!6} 1 & Outside & $ 1.2 $ & $ 600 $ \\
    2 & Outside & $ 1.225 $ & $ 400 $ \\
    \rowcolor{gray!6} 3 & Outside & $ 1.25 $ & $ 100 $ \\
    4 & Inside & $ 0.75 $ & $ 1 $\\
    \bottomrule
    \end{tabular}
    \caption{Chosen parameters for building classifiers, three of which tracing just outside of the observed exclusion limit in Figure \ref{fig:limits}, and one of which is completely inside the curve, following that of \cite{roxlo2018opening}. In doing so we observe how the mass difference between the stops and neutralinos affects the production and sensitivity of the simplified model.} 
    \label{tab:benchmarks}
\end{table}



%-------------------------------------------------------------------------%
\chapter{Methods to search for MSSM}
\label{chap:4}
%\section{Methods to search for MSSM}
In high energy physics, searches for new particles have relied on hadron colliders such as the LHC as they produce extremely high centre-of-mass\footnote{The Centre-of-Mass energy, denoted as $\sqrt{s}$, is a Lorentz invariant value such that $s=\Big( \sum\limits_{i=1}^2 E_i \Big)^2 - \Big( \sum\limits_{i=1}^2 \overrightarrow{p_i} \Big)^2 $. Note that $c=1$ and each $i$ represents the protons colliding.} energies through proton-proton collisions. These collisions produce heavy, typically extremely short-lived particles, including the top quarks and W-bosons. Searches beyond the SM proved to be a  challenge, or rather "unlucky", at such colliders, with no signs of supersymmetry since the discovery of the Higgs. In this section, we discuss briefly about the LHC and the CMS detector, understanding some important kinematic variables and how simulations are useful in the search for new physics in collider experiments.

%-------------------------------------------------------------------------%
\section{The Large Hadron Collider and CMS}
\label{sec:Detector}
Located on the border of France and Switzerland, the LHC is the current world-leading facility for collider experiments. The LHC currently operates at $ \sqrt{s}=13 \text{TeV} $, with an integrated luminosity\footnote{Luminosity is the number of events over a period of time, given a certain cross-section of the process. The integrated luminosity is the total luminosity overtime for, in the case of the LHC, the total collisions.} of $137\text{fb}^{-1}$ \cite{cms2019search}. One of the experiments in the LHC searching for supersymmetry is the Compact Muon Solenoid (CMS) collaborations \cite{chatrchyan2008cms} working independently to A Toroidal LHC ApparatuS (ATLAS) collaborations \cite{collaboration2008atlas}, the other experimental collaboration searching for supersymmetry.  To support any discoveries made at the LHC, each detector is constructed slightly differently. Detectors measure the energy, longitudinal and azimuthal components of the deposited particles. The azimuthal component, $\phi$, covers the range of $-\pi < \phi < \pi$. The longitudinal component is called the pseudorapidity, an angular measure of a particle in the detector relative to the direction of the beam axis given by
\begin{equation}
    |\eta|=-\ln\Big(\tan(\frac{\theta}{2})\Big)
    \label{eq:eta}
\end{equation}
Here, $\theta$ is the polar angle between the positive direction of the beam axis and the particle's three-momentum \textbf{p}, with the quantity preserving Lorentz invariance. The relationship between $\theta$ and $\eta$ is $\theta = \pm \pi/2 \rightarrow \eta = 0$, and $\theta = 0(\pi) \rightarrow \eta = +(-) \infty$. \\

The CMS detector \cite{chatrchyan2008cms} is a single large superconducting solenoid of 4T that aligns the protons traveling in a uniform direction within the detector. The solenoid is surrounded by several detectors with varying purpose as shown in Figure \ref{fig:detector}. Inside the solenoid exists an inner silicon track\footnote{Highly energetic charged particles leave a trail of ionised atoms, allowing highly precise and efficient measurements of the trajectories called \textit{tracks} \cite{chatrchyan2008cms}.}, electromagnetic calorimeters (ECALs) that cover the range $|\eta|<3.0$ and hadron calorimeters (HCALs) covering identical ranges. Additional forward calorimeters within the detector provide further coverage of up to $|\eta|=5$. Stable particles such as, but not limited to, the electron, photons, and protons are detected, with these particles propagating into and within the detector through particle \textit{showers}. Showers occur when the particles interact with atoms with high energies. Highly energetic muons, however, tend to escape the ECALs due to their larger mass compared to that of electrons. Thus muon chambers are required to detect outgoing muons in the form of ionisation, similarly to the inner tracks.  A quick summary of the functionality of the calorimeters is given in the dot-points below. 

\begin{figure}[htbp]
    \centering
    \includegraphics[width=\linewidth]{CMS_detector.png}
    \caption{A sliced view of the CMS detector, courtesy of \cite{ATLASandCMSDetector}. From left to right is the silicon track to trace the motion of the charged particles heading toward the subsequent layers of the ECALs and HCALs. The calorimeters are surrounded by the 4T superconducting solenoid that keeps the beam in a straight line. Finally, subsequent layers of muon chambers to detect the escaping muons surrounds the solenoid.}
    \label{fig:detector}
\end{figure}

\begin{itemize}
  \item \textbf{ECAL}: \par
  The ECALs measure electromagnetic showers that begin with highly energetic electrons and photons. The electrons entering the ECALs produce energetic photons in a process known as Bremsstrahlung, which then further produces a $e^{-}e^{+}$ pair. This process is repeated in the form of an electromagnetic shower until there is not enough energy remaining in the photons or electrons. The electrons and photons interact with atoms in the ECAL through ionisation and scintillation \cite{thomson2013modern}. The CMS detector uses lead-tungstate ($\text{PbWO}_4$) crystals due to its high density and short radiation length. The compact size allows the showers to be contained in a confined region \cite{chatrchyan2008cms}. 
  
  \item \textbf{HCAL}: \par
  The HCALs measure hadronic jets that originate from quarks produced in the collision, and the contribution of missing energy resulting from undetectable particles such as neutrinos and neutral long-lived hadrons. The nuclear interaction between the hadrons and atoms that form the HCAL layers prompt decays of these hadronic jets through ionisation, nuclear interactions, and strong interactions. These interactions form a cascade of interactions as showers, prompted by the alternating-layer structure of the HCAL \cite{thomson2013modern}. The CMS has focused on an alternating-layer structure with brass absorbers to prompt particle interactions, and plastic fluorescent scintillators to emit blue-violet light for detection \cite{chatrchyan2008cms}. 
\end{itemize}

It is possible to identify jets originating from $b$-quarks using a process known as \textit{b-tagging}. The relatively long lifetime of the $b$-quarks means the decays occur at a slight distance away from the collision point (primary vertex), producing a secondary vertex. At the secondary vertex, the $b$-jets form high-mass $B$ hadrons that can decay to highly energetic particles including semi-leptonic products. The tagging of $b$-jets exploit the long lifetime and resolving the secondary vertex, with the track information used to identify particles originating from $B$ hadrons \cite{collaboration_2013}. The identification of $b$-jets less efficient compared to identifying isolated leptons and remains a challenge to improve efficiency. 

%-------------------------------------------------------------------------%
\section{Kinematic variables in collider experiments}
The properties of the measured particles are recorded in the LHC data, with many such properties translated into the transverse plane of the beam axis. This is due to the boosting of many highly energetic particles along the beam axis, which makes it difficult to measure components such as their energy or momentum. Considering the beam direction to be the z-direction the transverse plane would be the $(x,y)$-plane, with the geometry of this depicted in Figure \ref{fig:beam}. \\

\begin{figure}[htbp]
    \centering
    \includegraphics[width=12cm, height= 6cm]{beam.png}
    \caption{The geometry of a collider experiment, with the beam axis considered as the z-direction and the transverse plane as the $(x,y)$-plane \cite{barr2011guide}. The pseudorapidity in Equation (\ref{eq:eta}) measures the polar angle and the azimuthal component $\phi$ is in the $(x,y)$-plane covering $-\pi \leq \phi \leq \pi$. }
    \label{fig:beam}
\end{figure}

One measure in the transverse plane is the transverse momentum, $p_T$, given by, 
\begin{equation}
    p_T = \abs{\overrightarrow{p_T}} = \abs{\sqrt{\overrightarrow{p_x}^2 + \overrightarrow{p_y}^2}}
    \label{eq:pt}
\end{equation}
where $p_x$ and $p_y$ are the momentum of particles in the $x$- and $y$-direction, respectively. It is possible to obtain certain quantities with $p_T$ such as the \textit{missing transverse energy} denoted as $\cancel{\it{E}}_{T}$, and the \textit{scalar transverse energy}\footnote{Otherwise known as the transverse hadronic energy sum.} denoted as $H_T$, given in Equation (\ref{eq:MET}) and Equation (\ref{eq:HT}), respectively. We will refer to the missing transverse energy as missing energy or MET, and the scalar transverse energy as scalar energy or HT from hereon. The missing energy is found by summing over the missing momentum from the detected particles as a consequence of energy and momentum conservation, and the scalar energy is given by the magnitude of the total momentum deposited by hadronic jets. \\
\begin{equation}
    \cancel{\it{E}}_{T} = \abs{\overrightarrow{\cancel{\it{E}}_{T}}} = \abs{- \sum_i \overrightarrow{p_T}(i)}
    \label{eq:MET}
\end{equation}
\begin{equation}
    H_T = \sum_i \abs{p_T(i)} 
    \label{eq:HT}
\end{equation}

In SM decays, leptonic final states of the top quarks produce neutrinos that freely pass through the detector, resulting in some $\cancel{\it{E}}_{T}$. Many MSSM events are predicted to have a large excess of $\cancel{\it{E}}_{T}$, more than the SM processes, due to a larger number of undetectable particles from the lightest neutralinos.  \\ 

%-------------------------------------------------------------------------%
\section{Simulations of a collider}
\label{sec:Sims}
The data collected in collider experiments do not contain labels associated with the physical processes measured. By simulating the process of interest closely following known theory and calculations, the existence, or lack of, the process of interest in the raw data can be identified by statistical techniques. There are typically three stages of simulation: the production/generation of the events through $pp$ collisions, the hadronisation of quarks leading to parton showers, and finally, the detection of electrons, muons, photons and jets as listed in Section \ref{sec:Detector}. \\

The generator level simulation of choice was MadGraph5\_aMC@NLO \cite{alwall2014automated} (MadGraph or MG5) due to its simplicity and fully-automated chain of simulations leading to the detector output. MadGraph calculates the cross-sections of the events of interest at leading-order (LO) by default\footnote{There is the capability for Next-to-leading-order (NLO) calculations as well, but we stick to the default LO calculations.}. However, the events do not mimic an output of the detector exactly and to achieve the complete simulation, these events must be processed further into what a real detector may observe. The intermediate steps such as hadronization of the quarks and their showers are governed by PYTHIA8.2 \cite{sjostrand2015introduction}, integrated into MadGraph, through Monte Carlo simulations. The output created by PYTHIA (HepMC files) may be fed directly into a detector level simulation of choice. \\

Delphes3 \cite{de2014delphes} is the detector level simulation chosen for this project due to its MG5 build-in feature, making the simulation process a single chain of operation. The CMS detector card was chosen to input parameters identical to selection criterion from CMS experiments \cite{cms2019search, cms2016searches, cms2017search}, shown in Table \ref{tab:efficiencies}. Jets are found through the FastJet finder \cite{cacciari2012fastjet} applying the anti-$k_t$ algorithm \cite{cacciari2008anti}. The output is a smeared calculation to the particle properties, in which isolated leptons, photons, jets, and $\cancel{\it{E}}_{T}$ are some of the main components reconstructed. Other information such as the track and vertices are also calculated. Muons are detected with an efficiency of 95\% while the electrons are detected with an efficiency of 85\%. Both leptons with a minimum $p_T$ of 20 GeV are considered to be isolated when it remains within a cone radius of $\Delta R <0.2$ and satisfies $I(l)<0.1$, given by Equation (\ref{eq:iso}) \cite{de2014delphes}. We require isolated leptons as they provide well-studied signatures from the SM that can be associated with new signatures from the weak interaction at high energies \cite{diaconu2004isolated}. The jets are identified when they have a minimum $p_T$ of 30GeV within a cone radius of $\Delta R <0.4$, with those originating from $b$-quarks are tagged with a lower efficiency of 60\%. \\

\begin{table}[htbp]
    \centering
    \begin{tabular}{c|c}
    \toprule
    Particle Type & Input parameters for the detector simulation \\
    \midrule
    \rowcolor{gray!6} Electrons & $\abs{\eta} < 1.442$, $p_T > 20\text{ GeV}$ @ $85\%$, $\Delta R <0.2$, $I(l) <0.1$\\
    Muons & $\abs{\eta} < 2.4$, $p_T > 20\text{ GeV}$ @ $95\%$, $\Delta R < 0.2$, $I(l) <0.1$\\
    \rowcolor{gray!6} Jets & $p_T>30$ GeV, $\Delta R = 0.4$\\
    b-tagging & $60\%$ \\
    \bottomrule
    \end{tabular}
    \caption{Chosen efficiencies for our detector simulation. The isolated electrons are 10\% less efficiently identified than isolated muons, also covering a narrower $\eta$ range. The isolation variable defined by Equation (\ref{eq:iso}) determines which charged electrons and muons from the detected particles are isolated, given a minimum $p_T$ of 20 GeV. The jets have a cone size of $\Delta R = 0.4$ travelling with $p_T>30$ GeV, where 60\% of those originating from $b$-quarks are tagged correctly.} 
    \label{tab:efficiencies}    
\end{table}

\begin{equation}
    \left. I(l) = \sum\limits_{i\ne l}^{R<\Delta R} p_T(i) \middle/ p_T(l) \right. \\
    \label{eq:iso}
\end{equation}


For this project, we stick to the properties listed in Table \ref{tab:variables}, including $H_T$, where leptons up to second-order and jets up to fourth order are selected. In addition, each of the jets will have an associated entry as to whether it is b-tagged or not. This information will be important to our analysis in the upcoming section. \\ 


\begin{table}[htbp]
    \centering
    \begin{tabular}{c|c|c|c} 
    \toprule
     & $\cancel{\it{E}}_{T}$ & $l^{\pm}_{i=1,2}$ & $j_{i=1,2,3,4}$ \\
    \midrule
    \rowcolor{gray!6} Energy & $\abs{\cancel{\it{E}}_{T}}$ & $ p_{T_l} $ & $ p_{T_j} $ \\
    $\eta$ & $\eta_{\cancel{\it{E}}_{T}}$ & $ \eta_l $ & $ \eta_j $ \\
    \rowcolor{gray!6} $\phi$ & $\phi_{\cancel{\it{E}}_{T}}$ & $ \phi_l $ & $ \phi_j $ \\
     %&  & $  $ & $  $\\
    \bottomrule
    \end{tabular}
    \caption{The variables extracted are the energy, $\phi$ and $\eta$ components of the MET, two charged leptons (one of which is a veto), and the four jets with a minimum of one $b$-tag were chosen.} 
    \label{tab:variables}
\end{table}

%-------------------------------------------------------------------------%
\section{Cut-based searches}
\label{sec:cut}
The cut-based analysis is an intuitive approach to the analysis of data from collider experiments, in which events are accepted or rejected based on physically motivated criteria. The objective of this method remains the same: to discriminate signal against background events and increase the sensitivity of the signal. An example diagram is shown in Figure \ref{fig:cut_flow}, where varying criterion (Recall Section \ref{sec:Detector}) such as $x>a$, $y>c$ and a multivariate variable $f(x,y)$, a combination of variables $x$ and $y$, accepts and rejects events. It results in the minimum number of background events remaining while maintaining a significant portion of signal events.  \\

\begin{figure}[htbp]
    \centering
    \includegraphics[width=0.5\linewidth]{cut_flow.png}
    \caption{A diagram depicting cut-flow analysis. For some criterion $x>a$, $y>c$ and a multivariate variable $f(x,y)$, the cuts accept events that satisfy such criteria, discarding events as background otherwise.}
    \label{fig:cut_flow}
\end{figure}

The benefit of cut-based searches is that these cuts are physically well-motivated and easy to keep track of. However, this may also be negative in certain scenarios where the physical properties of the particle in question are not as evident or well known. Searches for the process $\Tilde{t}\Tilde{t}^*  \rightarrow t\bar{t}\Tilde{\chi}_1^0\Tilde{\chi}_1^0$, but not limited to, have made use of a range of variables including $\cancel{\it{E}}_{T}$, the transverse mass $M_T$, the azimuthal difference between $\cancel{\it{E}}_{T}$ and a given jet, $\Delta \phi (\cancel{\it{E}}_{T}, \text{jet})$, and the number of specific particles within the data \cite{kraml2016scalar, chatrchyan2013search}. The cut-based analysis complements machine learning algorithms as a form of pre-selection technique that reduces background and signal events to similar signatures.

%-------------------------------------------------------------------------%
\chapter{Event generation - from pp collision to detection}
\label{chap:5}
%\section{Event generation - from the collision to detection}
%In this section, the method chosen to perform simulations for both the top and stop decays are shown. In addition, the preparation of data is explained to justify the usage of ML and their results.\\
%-------------------------------------------------------------------------%
\section{Background and Signal of interest}
\label{sec:production}
As discussed in Section \ref{sec:Sims}, MadGraph5 is used to perform simulations of particle colliders, simulating one million events for both signal and background events. At the generator level, we limit the missing energy $\cancel{\it{E}}_{T}$ to a minimum of 200 GeV to meet the pre-selection criteria listed in the following subsection. Our searches will be more effective by applying this criteria on the missing energy, as we push the sensitivity hence the efficiency of signal selection. The signal and background events differ significantly as shown in Figure \ref{fig:topMET}. Without this condition, the background events have a distribution much closer to zero with a mean of roughly 50 GeV. When constructing the classifiers and estimating efficiencies, the variation in MET affects the results. We can observe the background events distributed closer to that of the signal by requiring the signatures to be in a similar range, thereby pushing the efficiency and sensitivity of our results to a reasonable range. In addition, MG5 calculates a cross-section value associated with each process generated, something that is also required for our analysis. \\

\begin{figure}[htbp]
    \centering
    \includegraphics[width=\linewidth]{top-MET_new.png}
    \caption{A histogram of the distribution in generator-level and detector-level simulation in the top quark production $pp \rightarrow t\Bar{t} \rightarrow bW^{-}\Bar{b}W^{+} \rightarrow l^{\pm}\nu_l(\Bar{\nu}_l)q\Bar{q}$, created using MadAnalysis5 \cite{conte2013madanalysis, conte2014designing, dumont2015toward}. In the legend, the terms `GenLvl noCut' refers to the events plotted with the LHE file corresponding to the hard process alone without placing the requirement $\cancel{\it{E}}_{T}>200$ GeV. The lines represented with the term `w/Cut' corresponds to events simulated with this requirement. Furthermore, the detector effects on these events are those denoted with `DetLvl'. By placing the requirement of $\cancel{\it{E}}_{T}>200$ GeV we observe the missing energy for the background events (solid lines) shift closer to the distribution of the signal events (dashed lines). Note that the signal used in this plot is the first benchmark point with masses $m_{\Tilde{t}} = 1.2$ TeV and $m_{\Tilde{\chi}_1^0} = 600$ GeV.}
    \label{fig:topMET}
\end{figure}

Since the decay process involves both leptonic and hadronic particles as seen in Figure \ref{fig:topdecay}, the following were defined for the simulation. \\

\begin{lstlisting}[mathescape = true]
        define leptonic = l+ l- ta+ ta- vl vl$\sim$
        define hadronic = u c d s u$\sim$ c$\sim$ d$\sim$ s$\sim$ b b$\sim$
\end{lstlisting}

For the background process defined by Equation (\ref{eq:background}), the command to generate the events is given by
\begin{lstlisting}[mathescape = true]
            generate p p > t t$\sim$ , 
            (t > W+ b , W+ > leptonic leptonic), 
            (t$\sim$ > W- b$\sim$, W- > hadronic hadronic)
        
            add process p p > t t$\sim$ ,
            (t > W+ b , W+ > hadronic hadronic), 
            (t$\sim$ > W- b$\sim$, W- > leptonic leptonic)
\end{lstlisting}
where a diagram from one of its generated events can be seen in Figure \ref{fig:bkrdFeyn}. \\

Similarly, the process for signal\footnote{The parameter card was taken from \url{http://lpsc.in2p3.fr/projects-th/recasting/susy-vs-vlq/ttbarMET/} \cite{kraml2016scalar} under `SUSY-R', only adjusting the masses for stops and neutralinos in our runs.} production follows that of Equation (\ref{eq:signal}), in which the command for generating the events is given by
\begin{lstlisting}[mathescape = true]
        generate p p > t1 t1$\sim$ ,
        (t1 > t n1, (t > W+ b , W+ > leptonic leptonic)),
        (t1~ > t$\sim$ n1, (t$\sim$ > W- b$\sim$, W- > hadronic hadronic))
        
        add process p p > t1 t1$\sim$ , 
        (t1 > t n1, (t > W+ b , W+ > hadronic hadronic)), 
        (t1$\sim$ > t$\sim$ n1, (t$\sim$ > W- b$\sim$, W- > leptonic leptonic))
\end{lstlisting}
with an example diagram given in Figure \ref{fig:sigFeyn}. \\

\noindent\begin{minipage}{\textwidth}
\centering
  \begin{minipage}[htbp]{0.45\textwidth}
    \centering
    \includegraphics[width=\linewidth, keepaspectratio=true]{top_MG5.png}
    \captionof{figure}{Feynman diagram of the leading order background process $pp \rightarrow t \Bar{t} \rightarrow b\Bar{b}l^{+}jj\cancel{\it{E}}_{T} $.}
    \label{fig:bkrdFeyn}
  \end{minipage}
  \hfill
  \begin{minipage}[htbp]{0.45\textwidth}
    \centering
    \includegraphics[width=\linewidth, keepaspectratio=true]{stop_MG5.png}
    \captionof{figure}{Feynman diagram of the leading order signal process $ pp \rightarrow \Tilde{t}\Tilde{t^*} \rightarrow t \Bar{t} \chi^0_1\chi^0_1 \rightarrow b\Bar{b}l^{+}jj\cancel{\it{E}}_{T} $ where the final states are identical to that of the background in Figure \ref{fig:bkrdFeyn}.}
    \label{fig:sigFeyn}
  \end{minipage}
\end{minipage}
%-------------------------------------------------------------------------%
\section{Preselection}
Applying known conditions to searches reduces the total number of events that do not have desired signature, especially the background, allowing us to push the sensitivity of our searches for new particles. This process is known as \textit{pre-selection}, and it is a crucial step in our method. This process is identical to the cut-based analysis seen in Section \ref{sec:cut}, where placing certain cuts allows us to reduce the number of events constrained to the signature of interest. During the pre-selection process we require three conditions the data must meet: 
\begin{enumerate}
    \item $\cancel{\it{E}}_{T}>250$ GeV
    \item Only one charged lepton (No sign discrimination)
    \item A minimum of one $b$-tagged jet. The $b$-tagged jet with the highest $p_T$ is considered the only $b$-jet with the remainder considered as ordinary jets.
\end{enumerate}

Table \ref{tab:preselection} depicts the number of events remaining after each pre-selection criteria applied, from left to right. The disparity in the initial events stems from the fact that in the final count we need an equal amount of data to have a 50:50 split in our data between the signal and background events. This allows us to build our classifier effectively as most algorithms expect balanced data\footnote{By having a balanced data, the classifier is not biased into learning too much about one class.}, though there are solutions to this issue \cite{he2009learning}. In addition, Table \ref{tab:preselection} lists the cross-section, $\sigma$, of each process simulated with a corresponding Monte Carlo error estimate, which will be needed to determine the values discussed in the following section. It is shown that the cross-sections increase gradually as the mass difference between the stops and neutralinos grows larger, indicating that final states with more $\cancel{\it{E}}_{T}$ are less unusual, though not by a substantial amount. \\

\begin{table}[htbp]
    \centering
    \begin{tabular}{c|c|c|c|c||c} 
    \toprule
    Data & Initial & $\cancel{\it{E}}_{T}>250$ GeV & $1l^\pm$ & $1b$ & Cross-section, $\sigma$ (pb) \\
    \midrule
    \rowcolor{gray!6} Benchmark1 & 776800 & 559371 & 179472 & 101488 & $1.6\times10^{-4} \pm 6.7\times10^{-8}$ \\
    Benchmark2 & 758458 & 611119 & 187179 & 101488 & $4.0\times10^{-4} \pm  5.4\times10^{-7}$ \\
    \rowcolor{gray!6} Benchmark3 & 758498 & 643458 & 191944 & 101488 & $6.6\times10^{-4} \pm 2.7\times10^{-7}$ \\
    Benchmark4 & 818636 & 515694 & 172171 &101488  & $4.0\times10^{-3} \pm 1.6\times10^{-6}$ \\
    \rowcolor{gray!6} Background & $10^6$ & 357273 & 123933 & 101488 & $2.5 \pm 1.3\times10^{-3}$ \\
    \bottomrule
    \end{tabular}
    \caption{The number of events remaining at each step of the pre-selection process. Requiring that events satisfy a minimum of 250 GeV for $\cancel{\it{E}}_{T}$, that there is only one charged lepton and one $b$-tagged jet, the number of events significantly differs between simulated signal and background events. The initial number of events for the signal events was therefore reduced in order to create an equal division between the signal and the background within each data. Note that Monte Carlo errors are the errors associated with the cross-section values and not a complete uncertainty.} 
    \label{tab:preselection}
\end{table}



%-------------------------------------------------------------------------%
\chapter{Discriminating background and signal in the search regions}
\label{chap:6}
%\section{Discriminating background and signals with machine learning}
Machine learning is an increasingly popular data analytics technique commonly used in areas not limited to particle physics. It is useful in situations where a large amount of data produced or observed involves analysis that require better efficiency and computation time compared to traditional statistical techniques. In high energy physics, the use of machine learning has had a significant impact, for example, in the discovery of the Higgs boson using Boosted Decision Trees \cite{chatrchyan2012observation, aad2012observation, chen2015higgs}. \\


%-------------------------------------------------------------------------%
\section{Machine learning}
Machine learning  (ML) is an umbrella term for algorithms that ``learn" patterns from given data to give predictions through a process called \text{training}. It should be noted that there is a lower limit to the size of a dataset. For instance, for the number of entries $n$ and the number of variables $p$, if $p>n$ then there is not enough information for the classifier to train effectively\footnote{There are methods using regularization, for example, lasso, to deal with this issue \cite{james2013introduction}}. The preferred size for data is $p \ll n$ so that there is an abundance of information on which the classifier can train. There are two general categories in ML; \textit{supervised learning} and \textit{unsupervised learning}. Supervised learning requires data with labeled outcomes to build an effective classifier for the analysis of raw data, the results of which are unlabelled. Unsupervised techniques, on the other hand, are used when the data do not have labeled outcomes making them purely data-driven methods, unlike supervised techniques. Supervised techniques are commonly used in collider physics analysis as the simulation process allows as much data required to be generated, thus freely managing the size of data with labeled outcomes, resulting in the desired balanced data. \\

Training a classifier is relatively simple, with many choices available that are user-friendly, extremely versatile and deliver high performance. The classifiers are built by minimizing the selected error, otherwise known as a loss-function. The common loss-functions are simple metrics such as the \textit{Mean Squared Error (MSE)}, 
\begin{equation}
    MSE = \frac{1}{n} \sum\limits_{i=1}^{n}(y_i-\hat{y}(x_i))^2
    \label{eq:MSE}
\end{equation}
or the \textit{Root MSE (RMSE)},
\begin{equation}
    RMSE = \sqrt{\frac{1}{n}\sum\limits_{i=1}^{n}(y_i-\hat{y}(x_i))^2}
    \label{eq:RMSE}
\end{equation}
which perform well in general \cite{james2013introduction}. The error term is obtained by subtracting the predicted fit $\hat{y}_i$ at the $i$th observation with input parameters $x_i$ to the `true' fit $y_i$ \cite{james2013introduction}. Another high-performance loss-function is the negative log-likelihood function otherwise known as \textit{Log-Loss}
\begin{equation}
    l(x) = -\sum\limits_{i=1}^{n} \ln P(y_i| z_i) = \sum\limits_{i=1}^{n} \Big(-y_i z_i + \ln(1+e^{z_i}) \Big)
    \label{eq:logloss}
\end{equation}
for some regressor $z_i=\Tilde{\beta}^T x_i$ for a given $i$th observation in each $x_i$. The log-loss function is derived from the logistic function $g(z_i)=1/(1+e^{-z_i})$ and conditional probability. For a binary class problem with values of $y_i\in\{0,1\}$ the conditional probability is given by
\begin{equation}
    P(y_i| z_i) = g(z_i)^{y_i}(1-g(z_i))^{1-y_i}
\end{equation}
thereby simplifying to the logistic function for the background ($y_i=0$) and the inverse of $g(z_i)$ for the signal ($y_i=1$) \cite{knetml}. After some preliminary tests, including the three metric in Equations (\ref{eq:MSE})-(\ref{eq:logloss}), we found that the three metrics produced good results with the log-loss consistently performing the best on the basis of the accuracy of test classifiers. \\

When training a classifier, one must take care of \textit{overtraining}, a relatively common mistake. Overtraining\footnote{This is synonymous with overfitting data points with a higher-order polynomial to a given two-dimensional data, with the predictive power suffering as a consequence of overfitting.} happens when the parameters of the model are too restrictive in the model's ability to be flexible. The flexibility relates to how a model performs under various data of a similar nature. In other words, by limiting the model to perform well under a single dataset when training, we are limiting our ability to perform well in its predictive power in cases such as raw data from experiments. Overtraining is a mistake easy to prevent, and techniques to improve the performance of a classifier easily assure that this problem does not occur. The methods used in this project are \text{cross-validation} and \textit{hyperparameter grid-search}. \\

\begin{itemize}
    \item \textbf{Cross-Validation (CV)} \par
    Cross-validations (CV) is the task of partitioning training data to create a smaller set of training and test data and building smaller classifiers to perform tests \cite{james2013introduction}. The most common CV technique is the \textit{k-fold CV}, where the training data is randomly sampled into $k$ different subsets. The $k-1$ subsets are set as training data to build the classifier on, and the remaining subset is used to validate the classifier. This procedure is repeated $k$ times, where each classifier is scored on its accuracy. The overall result typically involves taking the $k$-fold average scores.  There are no strict guidelines for what the best value of $k$ is, but most ML users prefer to stick to either 5 or 10 because of its simplicity and performance \cite{james2013introduction}. \\
    
    \item \textbf{Hyperparameter Grid-Search} \par
    When building a classifier, it is difficult to find the parameters that boost its efficiency while being mindful of overtraining. A grid search can be done to solve this dilemma. A grid search utilizes CV at its foundations to test the efficiency of the selected parameters. By using this feature, several parameters can be tested with the best performing parameters that are used to train the original training data. The down-side to this, though, is that it is computationally expensive and multiple tests may even require some consistency in outcomes before some criteria are narrowed. \\
\end{itemize}

Fine-tuning parameters while cross-validating is not the only effective method for developing efficient classifiers. A common technique known as \textit{feature engineering} can be used to modify existing variables so that the accuracy of the prediction may improve the classifiers. A basic feature engineering was done in this project. It involved the rearrangement of the four jet entries so that the maximum $p _T$ jet with a $b$-tag is considered to be the $b$-jet originating from the top decay, and the following jets are ordered with $p_T$ values regardless of whether there is a $b$-tag or not. \\

The most popular algorithms used by the scientific community (not limited to physics) are Neural Networks (NNs) and Decision Trees (DTs) which work extremely well when used in the correct setting. In many communities, NNs are favoured over DTs. Nonetheless, multiple preliminary tests using the \textit{automl} feature from the \textit{h2o} package \cite{h2o} have shown that the tree-based method known as \textit{Extreme Gradient Boosting (EGB/XGBoost)} is most suitable for building our classifiers. In the chosen framework, which is R \cite{R}, the package with this available function is called \textit{xgboost}, whether available as a standalone use \cite{xgboost} or an extension of the h2o package \cite{h2o}.\\

%-------------------------------------------------------------------------%
\section{Tree-based methods and Extreme Gradient Boosting}
\label{sec:method}
We can intuitively think of tree-based methods as an extension to cut-based analysis (Recall Section \ref{sec:cut}). Instead of discarding selections that do not meet certain criteria, these selections may be further explored by the algorithm provided such selections exist. These splits are made by minimizing the chosen loss-function as discussed in the preceding section, making it difficult to infer the physical consequences of these splits. A simple diagram is depicted in Figure \ref{fig:tree}, where we see the criteria $x>a$ as our `root node' (the beginning of the tree), $y>c$, $f(x,y)$ and $g(x,y)$ as `decision nodes' (points that perform a split), and `branches' connecting the `terminal nodes'/`leaves' that is our prediction for the event signature. The mathematical formalism of a regression tree is described by
\begin{equation}
    T(X) = \sum_{i=1}^m c_i I(X\in R_i)
    \label{eq:DT}
\end{equation}
where $R_i$ is the partition of the predictor space corresponding to a leaf for $i=1,...,m$ non-overlapping regions. For each partition, $I(X\in R_i)$ is an indicator function that outputs predictions as one or zero given an observation $x_j\subseteq X$; either it is the signal or the background \cite{james2013introduction}. The regression tree creates an output between 0 and 1 that is weighted with some constant $c_i$. Using Figure \ref{fig:tree} as an example with $f(x,y)>b$ and $g(x,y)>d$ to predict a signal, the tree is formed as \\
\begin{equation}
    \begin{split}
        T(X) & = c_1I(x>a\cap y>c \cap f(x,y)>b)  
             + c_2I(x>a\cap y>c \cap f(x,y)<b)  \\
            & + c_3I(x>a\cap y<c) + c_4I(x<a \cap g(x,y)>d) + c_5I(x<a \cap g(x,y)<d)
    \end{split}
    \label{eq:DT_ex}
\end{equation}



\begin{figure}[htbp]
    \centering
    \includegraphics[width=10cm, height= 6cm]{DT.png}
    \caption{A diagram of a simple decision tree for a hypothetical signal selection following Figure \ref{fig:cut_flow} with an extra parameter $g(x,y)$ when $x<a$. The partition criteria in cut-based analysis were physically motivated and transparent, but the algorithm makes this process somewhat of a black-box, making direct physical interpretations challenging. The mathematical construction of this example tree is given by Equation (\ref{eq:DT_ex}).}
    \label{fig:tree}
\end{figure}

Most tree-based algorithms differ in the data sampling method and how the loss-function is minimized. In the case of EGB models, this is an extension to the \textit{Gradient Boosted Machine (GBM)} developed by Jerome Friedman \cite{friedman2001greedy}. \textit{Boosting} is an iterative algorithm in which an ensemble of weak models\footnote{A weak model consists of a small decision tree that is not an effective approximation to the data of interest.} are built sequentially, correcting its preceding model by re-weighting it with fitted errors from said model, leading to a final model that is highly representative of our data. Gradient boosting extends this idea with \textit{gradient descent}; an iterative algorithm that minimizes a differentiable loss-function, such as those in Equations (\ref{eq:MSE})-(\ref{eq:logloss}), to its local minima in its negative gradient direction specified by hyper-parameters \cite{gbmKaggle}. EGB/XGBoost adds to the features of regular GBMs with weighted quantile splittings and its ability to manage sparse\footnote{Since there are no missing entries in this project, the data is not sparse, or rather, it is `dense'.} data, incorporating parallel computing to achieve faster computation time compared to regular GBM algorithms \cite{chen2016xgboost}. We have been able to make the predictive power of our classifiers more accurate by considering certain hyper-parameters such as learning speed, depth and number of trees, and regularisation; a way to prevent overfitting by applying penalty terms, that can be tuned, to the loss-function  \cite{james2013introduction}. \\
%-------------------------------------------------------------------------%
\section{Metrics for model performance}
\label{sec:metrics}
%-------------------------------------------------------------------------%
\subsection{Confusion Matrix}
\label{sec:confMat}
The performance of a classifier can be summarized into a single table known as a confusion matrix. Displayed in Table \ref{tab:ConfMat}, a confusion matrix shows the distribution of correctly and incorrectly classified points predicted by the classifier. The correctly classified signal and background events are the True Positives (TP) and True Negatives (TN), respectively. Likewise, the incorrectly classified signal and background events are the False Negatives (FN) and False Positive (FP), respectively. The accuracy of the classifier is given by (TP+TN)/N and in a regression setting such as those in xgboost, a cut-off value between 0 and 1 varies this value. Choosing the ideal cut-off is important to maximize the Signal-to-Background Ratio (SBR)\footnote{The signal-to-background ratio is defined as the proportion of signal over the proportion of background events. In our analysis, we look to the cross-section and luminosity normalized TP and FP rates for signal and background respectively.} while maintaining high accuracy. \\

% Confusion matrix template
\begin{table}[htbp]
    \centering 
    \begin{tabu}{c|[2pt]c|c|c|c}
        \multicolumn{2}{c}{}&\multicolumn{2}{c}{Predicted}&\\
        \tabucline[2pt]{3-5}
        \multicolumn{2}{c|[2pt]}{}
        &\multicolumn{1}{c|}{Background} &\multicolumn{1}{c|[2pt]}{Signal} &\multicolumn{1}{c|[2pt]}{Total}\\
        \tabucline[2pt]{2-5}
        \multirow{\items}{*}{\rotatebox{90}{Simulated}}
        &\multicolumn{1}{c|[2pt]}{Background} & \multicolumn{1}{c|}{TN} & \multicolumn{1}{c|[2pt]}{FP} & \multicolumn{1}{c|[2pt]}{TN$+$FP} \\
        \cline{2-5}
        \multicolumn{1}{c|[2pt]}{}& \multicolumn{1}{c|[2pt]}{Signal} & \multicolumn{1}{c|}{FN} & \multicolumn{1}{c|[2pt]}{TP} & \multicolumn{1}{c|[2pt]}{FN$+$TP} \\
        \tabucline[2pt]{2-5}
        \multicolumn{1}{c|[2pt]}{} & \multicolumn{1}{c|[2pt]}{Total} & \multicolumn{1}{c|}{TN$+$FN} & \multicolumn{1}{c|[2pt]}{FP$+$TP} & \multicolumn{1}{c|[2pt]}{N}\\
        \tabucline[2pt]{2-5}
    \end{tabu}
    \caption{A confusion matrix for truth (simulated) and predicted labels and its components. The diagonal components are the correctly classified background (TN) and signal (TP) events. The off-diagonal components are the mis-classified background (FP) and signal (FN) events. }
    \label{tab:ConfMat}
\end{table}

%-------------------------------------------------------------------------%
\subsection{Approximate Median Significance}
\label{sec:AMS}
The statistical techniques introduced in Section \ref{sec:freqStat} requires calculating pdfs and test statistics that we did not have access to in the duration of this project. Instead, we turn to an evaluation metric presented in the Higgs Challenge \cite{adam-bourdarios_learning_2014} known as the \textit{approximate median significance (AMS)}, given by
\begin{equation}
    \text{AMS} = \sqrt{2\Big((s+b+b_r)\ln\Big(1+\frac{s}{b+b_r}\Big)-s\Big)}
    \label{eq:AMS}
\end{equation}
where $s$ and $b$ are the \textit{luminosity-normalized}\footnote{This includes the cross-section of the process.} TP and FP rates, respectively, and $b_r$ is a constant regularization term set at 10 introduced to not allowing $b$ approach zero, helping to reduce the variance of the AMS. We can obtain $s$ and $b$, by calculating
\begin{equation}
    s(b) = N_{s(b)}\times \epsilon_{s(b)} 
    \label{eq:N}
\end{equation}
where $N_{s(b)} = \sigma \int L(t) dt = \sigma \times 137$ fb$^{-1}$ \cite{thomson2013modern} is the number of expected events, and $\epsilon_{s(b)}=\text{(TP+FN)}_s/\text{N}$ $ \big(\text{(TN+FP)}_b/\text{N}\big)$  is the efficiency of the classifier given a dataset of size N. In our case, the test set, 1/3 of the combined data of roughly two hundred thousand events, was used to evaluate the AMS. The signal and background values when calculating the SBR we wish to maximize will refer to the value obtained via Equation (\ref{eq:N}) henceforth. \\

%The Gaussian significance discovery, discussed briefly in Section \ref{sec:freqStat}, with an estimated standard deviation\footnote{The standard deviation is actually given by $(n-\mu_b)/\sqrt{\mu_b}$ as the Poisson fluctuation of the background has a standard deviation of $\sqrt{\mu_b}$. Here, $n$ is the number of events and $\mu_b$ is the mean of the background. These numbers are replaced with their empirical counterparts: $s+b$ and $b$ respectively.} of $s/\sqrt{b}$ only hold when $s \ll b$ and $b\gg1$. This is not true in practice, therefore we turn to an approximate measure such as the AMS. 
%This measure is a derivation of the significance Z given by Equation (\ref{eq:Z})
%\begin{equation}
%    \text{Z} = \sqrt{2\Big( n\ln\Big(\frac{n}{\mu_b}\Big)-n+\mu_b\Big)}
%    \label{eq:Z}
%\end{equation}
%where $n$ is the number of events and $\mu_b$ is the mean of the background, requiring that $n>\mu_b$. The values $n$ and $\mu_b$ are replaced by $s+b$ and $b$ in Equation (\ref{eq:AMS}), respectively. Traditionally, the Z significane at $Z=5$ corresponds to a $p$-value of less than $2.9\times10^{-7}$, sufficient to claim a discovery \cite{adam-bourdarios_learning_2014}. However, the regularization term $b_r$ is equal to zero in this setting, thus differing this significance of values given by either equations. \\
%-------------------------------------------------------------------------%
\subsection{Feature Importance}
\label{sec:importance}

The simple tree diagram in Figure \ref{fig:tree} is not a practical tool to observe the structure and performance of the large number of trees created by algorithms such as xgboost. We then turn to \textit{feature importance} to quantitatively evaluate the performance of our classifiers with the variables used \cite{james2013introduction}. \\

XGBoost calculates feature importance in various metrics, one of which is the \textit{Gain} \cite{xgboost}. The `Gain' gives the contribution of each variable to the classifier as a fraction of the total gain from said variable upon splits. The gain of each split is defined as
\begin{equation}
    Gain = \frac{1}{2}\Bigg[ S_L + S_R + S_O \Bigg] -\gamma
    \label{eq:Gain}
\end{equation}
for scores on the new left(right) leaf $S_L$ ($S_R$) and the original leaf $S_O$. The penalty term $gamma$ regulates whether a split is created from a leaf i.e. if Equation (\ref{eq:Gain}) is negative, the leaf is not split \cite{xgboost_documentation}. 

%-------------------------------------------------------------------------%
\chapter{Observing the physics within our classifiers}
\label{chap:7}
%\section{Results}
%\begin{figure}[h!]
%    \centering
%    \includegraphics[width=14cm, height= 7.5cm]{.png}
%    \caption{}
%    \label{fig:}
%\end{figure}

%-------------------------------------------------------------------------%
\section{Performance of the classifiers}
%-------------------------------------------------------------------------%
%\subsection{Benchmark 1: \texorpdfstring{$\Tilde{t} = 1.2$}{ } TeV and \texorpdfstring{$\Tilde{\chi}_1^0 = 600$}{ } GeV}
An initial measure to visually understand how our classifiers have performed, a simple Receiver operating characteristic (ROC) curve can be constructed. A ROC curve shows the distribution of the predicted values given between 0 and 1, as a function of FP rate versus TP rate. This allows us to understand how much error we are willing to accept/reject for a given model. The ROC curve produced for all four benchmark data is presented in Figure \ref{fig:ROC}. It is not surprising that the benchmark point inside the exclusiong curve performed the worst amongst the four datasets. What is surprising, however, is that the third benchmark set which has the closest mass parameters to the fourth benchmark set performed the best although by a narrow margin. This indicates that a lighter $\Tilde{\chi}_1^0$ is favorable over heavier ones, though it is not conclusive due to the high performance of the other classifiers. \\


\begin{figure}[htbp]
    \centering
    \includegraphics[width=0.4\linewidth]{ROC_curve.png}
    \caption{ROC curve for all 4 benchmark points.}
    \label{fig:ROC}
\end{figure} 

Although the Area-under-the-curve (AUC) value is given in Figure \ref{fig:ROC}, it is maximized and does not directly represent the accuracy of our models. The accuracy of our models can be obtained by setting a cut-off to our predictions i.e. a number between 0 and 1. The closer the value is to 1 the lower the accuracy, but deliberately obtaining a high accuracy by setting a smaller value is also conter-intuitive as the Signal-to-Noise Ratio (SNR) will be smaller. Through various trials in cut-off values, it was determined for all four datasets that 0.6 is the optimal value to satisfy both high accuracy and high SNR. \\

For the benchmark point $\Tilde{t} = 1.2$ TeV and $\Tilde{\chi}_1^0 = 600$ GeV, the classifier performed quite well, producing a result of above $90\%$ accuracy. It is expected that the classifier performed less accurately when dealing with signal-like events. It correctly identified a smaller portion of true signal events and incorrectly classified more signal-like background events, as seen in Table \ref{tab:Values1}. The number of expected number of signal is significantly less than the expected number of background, producing an AMS value of 0.046. Considering the definition of AMS given in Section \ref{sec:metrics}, it supports the exclusion limit in Figure \ref{fig:limits}. \\

\noindent\begin{minipage}{\textwidth}
\centering
  \begin{minipage}[htbp]{0.65\textwidth}
    \centering
    \includegraphics[width=\linewidth]{bm1_distribution.png}
    \captionof{figure}{Distribution plot for predicted values.}
    \label{fig:dist_bm1}
  \end{minipage}
  \hfill
  \begin{minipage}[htbp]{0.34\textwidth}
        \centering
        \begin{tabular}{c|c} 
        \toprule
        Metric & Proportion \\
        \midrule
        \rowcolor{gray!6} TP & $43.9 \%$ \\
        TN & $48.4 \%$ \\
        \rowcolor{gray!6} FP & $6.2 \%$\\
        FN & $1.5 \%$ \\
        \rowcolor{gray!6} Accuracy & $92.3 \pm 0.2 \%$ \\
        \midrule
        $b$ & $4328$ \\
        \rowcolor{gray!6} $s$ & $3$ \\
        SBR & $0.07\%$\\
        \rowcolor{gray!6} AMS & $0.046$ \\
        \bottomrule
        \end{tabular}
        \captionof{table}{Values for parameter $\Tilde{t} = 1.2$ TeV and $\Tilde{\chi}_1^0 = 600$ GeV.} 
        \label{tab:Values1}
    \end{minipage}
\end{minipage}
%-------------------------------------------------------------------------%
%\subsection{Benchmark 2: \texorpdfstring{$\Tilde{t} = 1.225$}{ } TeV and \texorpdfstring{$\Tilde{\chi}_1^0 = 400$}{ } GeV}

\noindent\begin{minipage}{\textwidth}
\centering
  \begin{minipage}[htbp]{0.65\textwidth}
    \centering
    \includegraphics[width=\linewidth]{bm2_distribution.png}
    \captionof{figure}{Distribution plot for predicted values.}
    \label{fig:dist_bm2}
  \end{minipage}
  \hfill
  \begin{minipage}[htbp]{0.34\textwidth}
        \centering
        \begin{tabular}{c|c} 
        \toprule
        Metric & Proportion \\
        \midrule
        \rowcolor{gray!6} TP & $45.3 \%$ \\
        TN & $48.8 \%$ \\
        \rowcolor{gray!6} FP & $4.7 \%$\\
        FN & $1.2 \%$ \\
        \rowcolor{gray!6} Accuracy & $94.1 \pm 0.2 \%$ \\
        \midrule
        $b$ & $3250$ \\
        \rowcolor{gray!6} $s$ & $48$ \\
        SBR & $1.5\%$\\
        \rowcolor{gray!6} AMS & $0.84$ \\
        \bottomrule
        \end{tabular}
        \captionof{table}{Proportion of values of interest.} 
        \label{tab:Values2}
    \end{minipage}
\end{minipage}
%-------------------------------------------------------------------------%
%\subsection{Benchmark 3: \texorpdfstring{$\Tilde{t} = 1.25$}{ } TeV and \texorpdfstring{$\Tilde{\chi}_1^0 = 100$}{ } GeV}

\noindent\begin{minipage}{\textwidth}
\centering
  \begin{minipage}[htbp]{0.65\textwidth}
    \centering
    \includegraphics[width=\linewidth]{bm3_distribution.png}
    \captionof{figure}{Distribution plot for predicted values.}
    \label{fig:dist_bm3}
  \end{minipage}
  \hfill
  \begin{minipage}[htbp]{0.34\textwidth}
        \centering
        \begin{tabular}{c|c} 
        \toprule
        Metric & Proportion \\
        \midrule
        \rowcolor{gray!6} TP & $46.1 \%$ \\
        TN & $48.8 \%$ \\
        \rowcolor{gray!6} FP & $4.0 \%$\\
        FN & $1.1 \%$ \\
        \rowcolor{gray!6} Accuracy & $94.9 \pm 0.2 \%$ \\
        \midrule
        $b$ & $2798$ \\
        \rowcolor{gray!6} $s$ & $11$ \\
        SBR & $0.4\%$\\
        \rowcolor{gray!6} AMS & $0.21$ \\
        \bottomrule
        \end{tabular}
        \captionof{table}{Proportion of values of interest.} 
        \label{tab:Values3}
    \end{minipage}
\end{minipage}

%-------------------------------------------------------------------------%
%\subsection{Benchmark 4: \texorpdfstring{$\Tilde{t} = 750$}{ } GeV and \texorpdfstring{$\Tilde{\chi}_1^0 = 1$}{ } GeV}

\noindent\begin{minipage}{\textwidth}
\centering
  \begin{minipage}[htbp]{0.65\textwidth}
    \centering
    \includegraphics[width=\linewidth]{bm_In_distribution.png}
    \captionof{figure}{Distribution plot for predicted values.}
    \label{fig:dist_bm_in}
  \end{minipage}
  \hfill
  \begin{minipage}[htbp]{0.34\textwidth}
        \centering
        \begin{tabular}{c|c} 
        \toprule
        Metric & Proportion \\
        \midrule
        \rowcolor{gray!6} TP & $41.7 \%$ \\
        TN & $48.5 \%$ \\
        \rowcolor{gray!6} FP & $8.2 \%$\\
        FN & $1.6 \%$ \\
        \rowcolor{gray!6} Accuracy & $90.2 \pm 0.2 \%$ \\
        \midrule
        $b$ & $5727$ \\
        \rowcolor{gray!6} $s$ & $57$ \\
        SBR & $1\%$\\
        \rowcolor{gray!6} AMS & $0.75$ \\
        \bottomrule
        \end{tabular}
        \captionof{table}{Proportion of values of interest.} 
        \label{tab:Values_in}
    \end{minipage}
\end{minipage}

The AMS value of 0.75 obtained by this benchmark is less than but close to the value found by Roxlo and Reece \cite{roxlo2018opening} which was 1.72. This is a reassuring result, as the exclusion limit shows that indeed this point is unlikely to be the mass parameters for both the stop and the neutralino.

The result of the values may suggest that these regions will be soon excluded by the collider experiments???
%-------------------------------------------------------------------------%
\section{Data Visualisation using \textit{tourr}}

In this section, I would like to show how data visualisation could be useful in helping understand new physics beyond the SM better. By creating a two-dimensional projection of the data entailing of higher dimensions (i.e. many variables), we can observe the distribution of data points in such a projection. The \textit{tourr} package from R \cite{tourr} is an ideal program for such a task, producing interesting results. The projected tour requires an index to minimize the distance between certain points in a dataset, in which we chose to utilize the \textit{alpha-hull} index. A corresponding basis is generated with each projection, and the algorithm will search through potentially more optimal projections through each iteration. NEED TO EXPLAIN ABOUT ALPHA INDEX A BIT AND SHOW SOME SPLOTS AND TALK ABOUT THE PHYSICS WE CAN INTERPRET FROM IT.


  \begin{table}[htbp]
        \centering
        \begin{tabular}{c||c|c|c|c}
        \toprule
        &\multicolumn{1}{c|}{\bfseries Benchmark1}  &
        \multicolumn{1}{c|}{\bfseries Benchmark2}  &
        \multicolumn{1}{c|}{\bfseries Benchmark3} &
        \multicolumn{1}{c}{\bfseries Benchmark4} \\
        \midrule
        %----------------------------------%
        \textbf{Metric} & Proportion & Proportion & Proportion & Proportion \\
        \midrule
        \rowcolor{gray!6} TP & $42.0 \%$ & $43.8 \%$ & $44.9 \%$ & $39.3 \%$ \\
        TN & $49.4 \%$ & $49.5 \%$ & $49.4 \%$ & $49.5 \%$ \\
        \rowcolor{gray!6} FP & $0.6 \%$ & $0.5 \%$ & $0.6 \%$ & $0.5 \%$\\
        FN & $8.0 \%$ & $6.2 \%$ & $5.1 \%$ & $10.7 \%$ \\
        \rowcolor{gray!6} Accuracy & $91.3 \pm 0.2 \%$ & $93.3 \pm 0.2 \%$ & $94.2 \pm 0.2 \%$ & $88.8 \pm 0.2 \%$ \\
        \midrule
        $b$ & $403$ & $368$ & $371$ & $382$ \\
        \rowcolor{gray!6} $s$ & $2$ & $46$ & $11$ & $53$ \\
        SBR & $0.5\%$ & $12.5\%$ & $3.0\%$ & $13.9\%$\\
        \rowcolor{gray!6} AMS & $0.098$ & $2.32$ & $0.56$ & $2.62$ \\
        \bottomrule
        \end{tabular}
        \caption*{Proportion of important values obtained in the four benchmark parameters.}
    \end{table}


%-------------------------------------------------------------------------%
\chapter{Concluding Remarks}
%\section{Conclusion}
In this project, we explored the simplified model of the top squark decays, namely the process: $pp \rightarrow \Tilde{t}\Tilde{t^*} \rightarrow t \Bar{t} \Tilde{\chi^0_1}\Tilde{\chi^0_1} \rightarrow b\Bar{b}jjl\cancel{\it{E}}_{T}$ by simulating this process through MadGraph, Pythia and Delphes, across four different mass parameters for the top squarks and neutralinos. Three of which were sitting very close right outside the exclusion curve, in black, in Figure \ref{fig:limits}, and the other sitting well within the curve. \\

We used a machine learning algorithm known as xgboost to discriminate the signal and background events, followed by a simple statistical analysis using the approximate median significance (AMS). The three mass parameters outside the curve produce AMS values of 0.1, 0.31 and 0.6 suggesting that the background-only hypothesis for both discovery and upper limits cannot conclude definitively. In contrast, the data from within the exclusion curve resulted in an AMS value of 2.8 associated with a relatively low SBR of 15.7\%, given that the integrated luminosity is 137 $\text{fb}^{-1}$. This is a high AMS value, and by using a smaller luminosity value at 35.9 $\text{fb}^{-1}$, we observe this to reduce to AMS=1.37 with the SBR remaining relatively low at 15.6\%. This shows that indeed, this parameter $m_{\Tilde{t}} = 750$ GeV and $m_{\Tilde{\chi}_1^0} = 1$ GeV is excluded from searches. \\

We also turned to a high-dimensional data visualisation known as a guided tour to observe the patterns within the data as well as what qualities of the signal-like and background-like events made our classifiers perform poorly for such points. The most contributing variables were shown to be $\cancel{\it{E}}_{T}$ and the azimuthal component ($\phi$) of the $\cancel{\it{E}}_{T}$ and the charged leptons. The signal-like events showed a large spread in energy and angular dependence in most directions except for values closest to the $x$-direction orthogonal to the beam axis. The background-like events were distributed close to the $x$-direction orthogonal to the beam axis that we can imagine the charged lepton and missing energy to form a cone-like separation in the final states. As expected, the difference in energy for the events is the key contributor to differentiate signal and background. \\

In the future, it would be interesting to continue the analysis, particularly with the guided tour using more exotic variables such as the \textit{stransverse mass} to observe the contribution of such variables as we have done so well studied variables. We also hope to implement the guided tour as an analysis tool in helping us understand different processes for new exotic particles not limited to the simplified model in this project.  

\clearpage
%-------------------------------------------------------------------------%


%-------------------------------------------------------------------------%
%\newpage
\bibliographystyle{vancouver}
\bibliography{Thesis}

%-------------------------------------------------------------------------%


\end{document}
