\vspace{-0.5cm}
In this project, we explored the simplified model of the top squark decays, namely the process: $pp \rightarrow \Tilde{t}\Tilde{t^*} \rightarrow t \Bar{t} \Tilde{\chi^0_1}\Tilde{\chi^0_1} \rightarrow b\Bar{b}jjl\cancel{\it{E}}_{T}$ by simulating this process through MadGraph, Pythia and Delphes, for four different mass parameters for the top squarks and neutralinos. Three were sitting just outside the exclusion curve (demarcated in black in Figure \ref{fig:limits}) and the other sitting well within the curve. \\

We used a machine learning algorithm known as xgboost to discriminate the signal and background events, followed by a simple statistical analysis using the approximate median significance (AMS). The three mass parameters outside the curve produce AMS values of 0.1, 0.31 and 0.6 suggesting that the background-only hypothesis for both discovery and upper limits is expected to be inconclusive. In contrast, the parameters from within the exclusion curve resulted in an AMS value of 2.8 associated with a relatively low SBR of 15.7\%, given that the integrated luminosity is 137 $\text{fb}^{-1}$. This is a high AMS value, and by using a smaller luminosity value at 35.9 $\text{fb}^{-1}$, we observe this to reduce to AMS=1.37 with the SBR remaining relatively low at 15.6\%. This shows that indeed, this parameter $m_{\Tilde{t}} = 750$ GeV and $m_{\Tilde{\chi}_1^0} = 1$ GeV is also excluded from searches. \\

We also turned to a high-dimensional data visualisation known as a guided tour to observe the patterns within the data as well as what qualities of the signal-like and background-like events made our classifiers perform poorly for such points. The most contributing variables were shown to be $\cancel{\it{E}}_{T}$ and the azimuthal component ($\phi$) of the $\cancel{\it{E}}_{T}$ and the charged leptons. The signal-like events showed a large spread in energy and angular dependence in most directions except for values closest to the $x$-direction orthogonal to the beam axis. The background-like events were distributed close to the $x$-direction orthogonal to the beam axis, that we can imagine the charged lepton and missing energy to have a narrow separation in the final states. This was further shown in a distribution plot for $\Delta \phi(\cancel{\it{E}}_{T},l)$ with a narrow distribution for the background events and broader distribution for the signal events. As expected, the difference in energy for the events remained the main contributor to differentiate signal and background. \\

In future, we can improve the methods in our analysis by incorporating the new $\Delta \phi(\cancel{\it{E}}_{T},l)$ variable, and other physically motivated variables seen in cut-based searches $\Delta \phi(\cancel{\it{E}}_{T},l)$. We can also perform simulations in MadGraph with Next-to-leading order (NLO) calculations instead of LO for more accurate kinematics. Furthermore, with the guided tour showing an important relationship that is $\Delta \phi(\cancel{\it{E}}_{T},l)$, we showed that data visualisation with a guided tour is an effective tool for data analysis, hence it is recommended for use in other searches as well.