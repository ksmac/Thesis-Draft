The discovery of the Higgs boson at the Large Hadron Collider in 2012 \cite{chatrchyan2012observation,aad2012observation} completed the Standard Model of particle physics, with fundamental particles and the force carriers mediating interactions of these particles. These particles and their interactions are now studied with high precision, yet many properties of the universe remain unexplained by the Standard Model, one of which is dark matter. Regular matter in the universe constitutes only 5\% of the total density, whereas dark matter constitutes roughly 23\% \cite{thomson2013modern}. This suggests that we only understand a small fraction of the physics that explains our universe.  \\

There are many extensions to the Standard Model that seek to describe physics beyond the Standard Model. Supersymmetry is a concept that is included in many models attempting to explain new physics, one realisation of which is the Minimal Supersymmetric Standard Model \cite{martin1997supersymmetry}. In addition to extra Standard Model-like Higgs, this model introduces partners to the Standard Model particle content called superpartners, so that the particle content appears duplicated, conveniently including new particles that satisfy the supposed properties of dark matter. The mixture of the superpartners of the neutral Standard Model force carriers provides us with four \textit{neutralinos} in which the lightest one resembles the properties of dark matter. This lightest neutralino is thought to be the lightest supersymmetric particle and is stable \cite{martin1997supersymmetry}. In this project, we considered a simplified model from amongst many candidate decays in the Minimal Supersymmetric Standard Model. The decay process of interest is $\Tilde{t}\rightarrow t\Tilde{\chi}_1^0$ where the supersymmetric top squark ($\Tilde{t}$) produces the Standard Model top quark ($t$) and the lightest neutralino ($\Tilde{\chi}_1^0$), as the top squarks are thought to be the next-lightest supersymmetric particle. \\

The success of the Higgs discovery is credited to theory, and the statistical techniques and analytical tools used in collider experiments. The searches for supersymmetric particles follow the same methods, in which experiments have been gradually constraining the limits for the masses of the top squarks and neutralinos. Simulations and Machine Learning are some of the key aspects in analyzing experimental data \cite{chatrchyan2012observation,aad2012observation}, and so we follow similar techniques in our method. Furthermore, we can visualize the data with a guided tour, a form of data projection, in the hope of getting an insight into the physics enfolding with these particles. \\

\newpage
\noindent\textbf{Outline:}\\
In this thesis, we present a brief introduction to the Standard Model and the Minimal Supersymmetric Standard Model in Chapter \ref{chap:2}, followed by the details of the top squark, its production, and how experiments search for new particles in Chapter \ref{chap:3}. Chapter \ref{chap:4} presents the tools and supporting arguments related to our method, particularly in how particle detectors work and how simulations are performed to assist in searches. Our methods for simulation and data creation are discussed in Chapter \ref{chap:5}. The methods for analysis, particularly machine learning and the applied statistical techniques, are explored in Chapter \ref{chap:6}. We present our results in Chapter \ref{chap:7} with the performance of our classifiers, and exploring the data with data visualisation using a guided tour.