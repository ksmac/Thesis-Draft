The discovery of the Higgs boson in 2012 at the Large Hadron Collider \cite{chatrchyan2012observation,aad2012observation} completed the Standard Model of particle physics, with fundamental particles and the force carriers mediating interactions of these particles. These particles and their interactions are now studied with high precision, however, properties of the universe unexplained by the Standard Model exits, one of which is dark matter. Regular matter in the universe constitutes of only 5\% of the total density, whereas dark matter constitutes of roughly 23\% \cite{thomson2013modern}. This variation alone suggests that we only understand a fraction of the physics that explains our universe.  \\

Many extensions to the Standard Model exist to explain physics beyond the Standarad Model. Supersymmetry is a concept that is included in many models attempting to explain new physics, one of which is the Minimal Supersymmetric Standard Model \cite{martin1997supersymmetry}. This model introduces superpartners to the Standard Model particle content in addition to extra Higgs, such that it looks `exactly duplicated', conveniently including new particles that satisfy properties assumed for dark matter. The mixture of the superpartners of the neutral Standard Model force carriers provide us with four \textit{neutralinos} in which the lightest one resembles the properties of dark matter. This lightest neutralino is thought to be the lightest supersymmetric particle and is stable \cite{martin1997supersymmetry}. In this project, we considered a simplified model from amongst many candidate decays in the Minimal Supersymmetric Standard Model. The decay process of interest is $\Tilde{t}\rightarrow t\Tilde{\chi}_1^0$ where the supersymmetric top squark ($\Tilde{t}$) produces the Standard Model top quark ($t$) and the lightest neutralino ($\Tilde{\chi}_1^0$), as the top squarks are thought to be the next-lightest supersymmetric particle. \\

The success of the Higgs discovery is credited to theory, and statistical techniques and analysis tools used in collider experiments. Searches for supersymmetric particles have followed the same procedures, where experiments have been gradually pushing the limits to for masses to the top squarks and neutralinos. Simulations and Machine Learning are some key aspects in analyzing experimental data and so we follow similar techniques in our method. Furthermore, we can visualize the data with a guided tour in hopes to get an insight into the physics enfolding with these particles. \\

\newpage
\noindent\textbf{Outline:}\\
We present a brief introduction to the Standard Model and the Minimal Supersymmetric Standard Model in Chapter \ref{chap:2}, followed by the details of the top squark, its production, and how searches for this particle are performed experiments in Chapter \ref{chap:3}. Chapter \ref{chap:4} presents the tools and supporting arguments related to our method, particularly in how particle detectors work and how simulations are performed to assist in searches. Our methods for simulation and the creation of the datasets are discussed in Chapter \ref{chap:5}. The method for analysis, particularly machine learning and the applied statistical techniques, are explored in Chapter \ref{chap:6}. We present our results in Chapter \ref{chap:7} with the performance our classifiers and data visualisation with a guided tour.