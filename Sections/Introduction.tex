The discovery of the Higgs boson in 2012 at the Large Hadron Collider completed the Standard Model of particle physics to the extent we see today, with fundamental particles and the force carriers mediating interactions of these particles are now studied with high precision. There are, however, properties of the universe unexplained by the Standard Model, namely gravity and dark matter. Regular matter in the universe consist of only 5\% \\

We present a brief outline to the Standard Model and the Minimal Supersymmetric Standard Model in Chapter \ref{chap:2}, followed by the details in the top squark and its production in Chapter \ref{chap:3} in how searches for such exotic particles are performed. Chapter \ref{chap:4} presents ideas related to our method, particularly in how particle detectors work and how performing simulations allow us to push the sensitivity of the searches. Our methods for simulation and the creation of the datasets are discussed in Chapter \ref{chap:5}.