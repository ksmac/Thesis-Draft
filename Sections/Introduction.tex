The discovery of the Higgs boson in 2012 at the Large Hadron Collider completed the Standard Model of particle physics to the extent we see today, with fundamental particles and the force carriers mediating interactions of these particles are now studied with high precision. There are, however, properties of the universe unexplained by the Standard Model, namely gravity and dark matter. Regular matter in the universe consist of only 5\% of the total density, whereas dark matter consists of roughly 23\% \cite{thomson2013modern}. This variation alone suggests that we only have a fraction of the physics that explains how our universe understood.  \\

An extension to the Standard Model known as Supersymmetry conveniently includes new particles that satisfy properties assumed for dark matter. Supersymmetry has some variations of which one of the models is the Minimal supersymmetric Standard Model \cite{martin1997supersymmetry}. This model introduces superpartners to the Standard Model particle content such that it looks `exactly duplicated'. The mixture of superpartners to the neutral Standard Model force carriers provide us with four \textit{neutralinos} in which the lightest one resembles the properties of dark matter. This lightest neutralino is also theorized to be the lightest supersymmetric particle as well, hence unable to decay or has an extremely long lifetime that is considered to be stable \cite{martin1997supersymmetry}. In this project, we considered a simplified model from amongst many candidate decays in the Minimal Supersymmetric Standard Model. The decay process of interest is $\Tilde{t}\rightarrow t\Tilde{\chi}_1^0$ where the supersymmetric top squark ($\Tilde{t}$) produces the Standard Model top quark ($t$) and the lightest neutralino ($\Tilde{\chi}_1^0$), as the top squarks are theorized to be the next-lightest supersymmetric particle. \\

The success of the Higgs discovery is attributed to theory as well as statistical techniques and analysis tools used in collider experiments. Searches for supersymmetric particles have followed the same procedures, where experiments have been gradually pushing the limits to potential masses for these new exotic particles. Simulations and Machine Learning are some key aspects in analysis experimental data and so we follow similar techniques in our method. Furthermore, we can visualize the data with a form of projection pursuit known as a guided tour in hopes to get an insight into the physics enfolding with these exotic particles. \\

\noindent\textbf{Outline:}\par
We present a brief introduction to the Standard Model and the Minimal Supersymmetric Standard Model in Chapter \ref{chap:2}, followed by the details in the top squark and its production in Chapter \ref{chap:3} in how searches for such exotic particles are performed. Chapter \ref{chap:4} presents ideas related to our method, particularly in how particle detectors work and how performing simulations allow us to push the sensitivity of the searches. Our methods for simulation and the creation of the datasets are discussed in Chapter \ref{chap:5}. The method for analysis, particularly machine learning and applied statistical techniques, is explored in Chapter \ref{chap:6}, followed by the results from building our classifiers with some statistical analysis and data visualisation presented in Chapter \ref{chap:7}.